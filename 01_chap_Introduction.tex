\chapter{Introduction}

In 2019 it was estimated there were 3.4 billion smartphone users worldwide \cite{SmartphoneUsers2019}.  Smartphones are becoming deeply integrated into our everyday lives.  We use them to navigate, hail taxis, order goods online, and even pay for goods using contactless payment.  Of those 3.4 billion smartphones users, 1.6 billion were using Android devices \cite{AndroidUsers2019}.  But, the use of Android goes beyond just smartphones.  Android is a popular O/S for smart TVs, tablets, automobile infotainment, smartspeakers, smartwatches, and multitudes of other IoT devices.

The threat of malware increases as we adopt these devices.  McAfee has claimed security attacks on Mobile devices are so prevalent that in 2020 they opened their threat report with the headline ``Mobile Malware Is Playing Hide and Steal'' \cite{McAfeeMobileThreatReport}.  Android is so prevalent that security gaps and exploitations can affect massive groups of people.  When a person installs an application from an app store, how do they know it is not performing some secret activity?\\
\\
\noindent How can a person be certain that their smartphone is not eavesdropping on conversations?\\
\noindent How can a person be sure the camera on their phone is not surreptitiously sending images to an unknown observer?\\
\\
For the sake of peoples privacy and the prevention of crime, the user must always be aware when software is using privacy-sensitive features of their device.

Android security prevents applications from accessing sensitive features of the device if the user has not permitted that.  But, evidence shows that applications can collude to circumvent this security system \cite{PrologAppCollusion}:  If an application with permission to read the user's contacts passes that information to another application with permission to communicate with an outside party, then the two are colluding with one another to expose the user's contacts without the user's permission and knowledge.

\section{Project Aims}

Our objective is to use a program validation technique called runtime verification to monitor the device for security attacks.  We will produce an app that will monitor activity on Android devices and warn users if a potential security breach has happened.

\begin{itemize}
\item It should be possible to install the application on an Android device.
\item The application should be capable of detecting security breaches involving any other applications without modification to those other applications.
\item The impact of the application on the device's performance will be minimal.
\item It should be possible to detect a security breach in realtime.
\item It is not required that the user can enter security properties themselves.
\item We leave the concrete form of warning open and concentrate on recognising security breaches only.
\end{itemize}

Part of the challenge is that monitoring increases the use of hardware resources and Android runs on devices with limited processor and power resources.  It is unclear if runtime verification can be used on Android while maintaining adequate performance.  We will answer this question.

\section{Pegasus}

%Pegasus spyware was discovered in 2016 and recently exposed in the mainstream media \cite{PegasusGuardian}.  The claim is that Pegasus has infected at least 50,000 devices belonging to people in government and journalism.  What we know about Pegasus on Android\cite{PegasusOnAndroid}:

Pegasus spyware was discovered in 2016.  A mainstream media expose\'e \cite{PegasusGuardian} claims Pegasus has infected at least 1400 devices belonging to people in government and journalism.  What we know about Pegasus on Android\cite{PegasusOnAndroid}:

\begin{itemize}
\item After the infected device boots, Pegasus looks for configuration settings by parsing query string parameters from a URL in the browser's history or reading a file present on the device.
\item It gathers information by querying the databases of messaging and communication apps.
\item It can actively take images from the camera.
\item It will eavesdrop on the user by answering calls from specific numbers without indicating a call is in progress.
\item It sends information to an attacker via a network connection.
\end{itemize}

In the context of Pegasus, we speculate that our runtime verification approach could successfully detect at least some Pegasus attacks:  We reason that threats require the use of the operating system to access resources because Android runs on a diverse array of hardware, and the O/S is the only unified means to access resources.  Consequently, the Pegasus spyware works at the operating system level.  We will warn the user if any security-sensitive resource gets used without user interaction.

\section{Presentation}

On the 18th of May 2021, we presented our work to a workshop conference for cyber-physical systems (2021 CPS-IOT).  The conference was hosted by Vanderbilt University and sponsored by the Association for Computing Machinery, IEEE, NSF, and SIGBED.  An abstract got submitted, peer-reviewed and is on the conference website at \url{https://sites.google.com/virginia.edu/mt-cps2021/program}.  We received positive feedback and some valuable points on our presentation.

\section{Overview}

\noindent Starting with chapter \ref{chap:Collusion} we establish the concept of collusion.  The villain of this story is malware who, in this case, steals information from the Android user.  We refer to prior studies that demonstrate colluding applications can circumvent the security model.\\

\noindent Chapter \ref{chap:Runtime Verification} proposes that a software validation technique called Runtime Verification can solve the collusion threat. We first explain the technique and then put its distinguishing features into context with other validation techniques. Then we examine some challenges of using runtime verification to detect collusion.\\

\noindent Linear Temporal Logic goes hand-in-hand with runtime verification and is the subject we introduce in chapter \ref{chap:Linear Temporal Logic}.  We cover the concept of a trace of events, followed by the grammar, and semantics of LTL.\\

\noindent Chapter \ref{chap:Rosu-Havelund Algorithm} looks at an algorithm developed by \GRKH\ for evaluating an LTL formula over a trace with little computational cost.  We explain the algorithm using examples that walk through the evaluation of a formula over a trace step-by-step.\\

\noindent In chapter \ref{chap:Android Platform} we go into detail about the Android platform architecture, the security model, and communication provisions.\\

\noindent Next, we require a means to capture the activities of applications running on Android. Chapter \ref{chap:Xposed Framework} presents a third-party framework called Xposed for this purpose.  We go into what it does, its origins, and how the framework gives us the ability to log operating system calls made by any application running on Android.  In this chapter, we also describe how to set up the development environment we used.\\

\noindent Chapter \ref{chap:Related Work} puts the project into context with other related work.  We discuss other approaches to detecting collusion and other approaches to Android security in general.\\

\noindent Our contribution begins with chapter \ref{chap:Monitoring For Security Properties} where we introduce a formula for detecting collusion, discuss some the hazards encountered when developing a monitor, strategies for overcoming them and how we might respond to a security breach.\\

\noindent Chapter \ref{chap:Monitoring System Architecture} progresses to outline the software architecture of our monitoring system, why we have taken this structure, and its benefits.  Then we give some advice on development with the Xposed framework based on experiences acquired during this project.\\

\noindent We investigate the suitability of the \RH\ algorithm to realtime monitoring in chapter \ref{chap:Runtime Verification with the Rosu-Havelund Algorithm}.  We start with a complexity analysis of the algorithm then seek to validate the outcome by performing practical tests on an implementation.  The results lead us to a discussion of the algorithm's scalability and a conclusion about its suitability to this application.\\

\noindent Following our discoveries regarding the \RH\ algorithm, we propose a variation in chapter \ref{chap:Reverse Rosu-Havelund Algorithm}.  We implement the new algorithm, carry out similar practical tests, and compare the results to those of the standard algorithm.\\

\noindent In the final chapter of the contribution, chapter \ref{chap:MonitorInAction}, we bring together everything from the design and implementation into a demonstration of our monitor detecting collusion between two applications running on a real Android device.\\

\noindent Chapter \ref{chap:Summary} summarises the project and we discuss where we might go next in chapter \ref{chap:Future Work}.

