\chapter{Runtime Verification with the \RRH\ Algorithm}
\label{chap:Reverse Rosu-Havelund Algorithm}

We address the shortcomings of the standard algorithm that we discovered in Chapter \ref{chap:Runtime Verification with the Rosu-Havelund Algorithm} by proposing a novel algorithm that we call \RRH.  It is based upon the original but with a simple modification: We change the order in which the trace events get visited during evaluation.  The order gets reversed to evaluate the trace from the earliest entry to the latest.  The effect of this is that when a new event occurs and the trace must be re-evaluated, the new evaluation always begins at the same event as the previous evaluation; the earliest event.

%The reverse algorithm solves the shortcomings of the standard algorithm by changing the order in which the trace events are visited during evaluation.  The order is reversed to evaluate the trace from the earliest entry to the latest.  The effect of this is that when a new event occurs and the trace must be re-evaluated, that new evaluation always begins at the same event as the previous evaluation; the earliest event.

This is in contrast to the standard algorithm which visits the latest event first and moves backwards to the earliest event.  The problem is that the latest event is always different after a new event has occurred, so when evaluating a trace under the standard algorithm, that evaluation always begins from a new event.  This is a crucial detail because the standard algorithm involves the \textit{next} register.  The purpose of the \textit{next} register is to accumulate the evaluation result of each event in the trace as they are visited.  But at the moment when a new event occurs the result held in the \textit{next} register becomes invalid because it was determined before the new event happened.  Therefore, under the standard algorithm the \textit{next} register must be entirely redetermined every time a new event occurs, otherwise the final result of evaluating the trace will be incorrect.\\
\\
\section{Comparison Between Standard and Reverse Algorithms When Evaluating an Extending Trace}

To demonstrate the problem with the standard algorithm and the solution, we will compare the intermediate result of evaluating a trace that increases in length.  We will first use the standard, and then the reverse algorithm and show the intermediate result after each evaluation.  

\begin{myEx} Standard \RH\\
\\
\noindent
The \textit{next} register is defined as an intermediate result in a computational sequence that we wish to reuse between evaluations.\\
\\
\indent Let $t_1$ be a trace,\\
\indent as events occur $t_1$ is extended by one event,\\
\indent after the first event $t_1 =\ <e_1>$,\\
\indent after the second event $t_1 =\ <e_1, e_2>$,\\
\indent after n events $t_1 =\ <e_1 ... e_n>$.\\
\\
The computation sequence obtained when the $n^{th}$ event has occurred has a step for each suffix in the trace because the algorithm iterates from the last entry to the first:\\
\\$
\indent next(e_n),\\
\indent next(e_{n-1}, e_n),\\
\indent ...\\
\indent next(e_1 ... e_n)\\
\\
\indent intermediate\ result = next(e_1 ... e_n)\\
$\\
The result of the evaluation is the first element in the \textit{next} register.  The intermediate result is the \textit{next} register - $next(e_1 ... e_n)$.\\
\\
When a further event occurs, the trace is extended to $t_2 =\ <e_1 ... e_n, e_{n+1}>$.  The computational sequence obtained from evaluating the  $t_2$ trace is:\\
\\$
\indent next(e_{n+1}),\\
\indent next(e_n, e_{n+1}),\\
\indent ...\\
\indent next(e_1 ... e_n, e_{n+1})\\
\\
\indent intermediate\ result = next(e_1 ... e_n, e_{n+1})\\
$\\
None of the steps performed to compute the intermediate result from $t_1$ contain $e_{n+1}$, yet all the computation steps performed to evaluate $t_2$ do.  There is no overlap between the computational sequences of the two evaluations thus it is impossible to reuse the \textit{next} register between evaluations.  Instead the register must be entirely recomputed after every new event so that it includes the evaluation of the $e_{n+1}$ event.
\\
\qed
\end{myEx}

To address this problem we rename the \textit{next} register to \textit{previous} to reflect the fact that it now stores the accumulated evaluations of all earlier events in the trace rather than later events.  Because all evaluations under the reverse algorithm start from the same event it means that even after a new event occurs the \textit{previous} register still holds a valid accumulation of all previous events.  Therefore it is not necessary to redetermine the \textit{previous} register every time a new event occurs.

\begin{myEx}\RRH\\
\\
Let $t_3$ be a trace of n events, $t_3 =\ <e_1 ... e_n>$.  The computation sequence we get when the $n^{th}$ event occurs has a step for each prefix of the trace because the reverse algorithm iterates from the first entry to the last:\\
\\$
\indent previous(e_1),\\
\indent previous(e_1, e_2),\\
\indent ...\\
\indent previous(e_1 ... e_n)\\
\\
\indent intermediate\ result = previous(e_1 ... e_n)$\\
\\
\noindent 
When a further event occurs, the trace is extended to $t_4 =\ <e_1 ... e_n, e_{n+1}>$.  The computational sequence we obtain is:\\
\\$
\indent previous(e_1),\\
\indent previous(e_1, e_2),\\
\indent ...\\
\indent previous(e_1 ... e_n),\\
\indent previous(e_1 ... e_n, e_{n+1})\\
\\
\indent intermediate\ result = previous(e_1 ... e_n, e_{n+1})\\
$\\
We can see the computational sequences do overlap.  The only step that was performed for $t_4$ and not for $t_3$ is the last step, $previous(e_1 ... e_n, e_{n+1})$, that corresponds to the new event.  This means the intermediate result of evaluating $t_3$ is the same result we arrive at after all but the final computational step for $t_4$.  Therefore we can reuse the \textit{previous} register between evaluations without recomputing it.  The only computational step that must get performed to evaluate $t_4$ is the final step that evaluates the formula over the new event.\\
\qed
\end{myEx}

\section{Consequences of Reverse Evaluation}
\label{sec:ReverseCollusionFormula}

When we make these changes to the algorithm, the effect is that we do not need to visit every event in the trace whenever a new event occurs.  The \textit{previous} register already holds the cumulative outcome of evaluating all earlier events in the trace, so if the evaluation of a new event calls for us to refer to the evaluation result of a previous event then we can refer the \textit{previous} register without additional computational cost.  Therefore, whenever an event occurs, only that single event has to be visited rather than the whole trace.

While this is a substantial complexity improvement over the standard algorithm, it comes at an expense.  A problem arises when evaluating the future LTL operators; until, eventually, and next.  The semantics of the future LTL operators refer to the evaluations of events following the event being evaluated.  But, when the latest event is being evaluated, the future events are not known.  It is therefore impossible to evaluate formulae that include future LTL operators.

This is not a disaster because having the \textit{previous} register at hand means it is possible to evaluate LTL formulae written in terms of past events.  The reverse algorithm replaces the future LTL operators defined in sections \ref{sec:LTLFutureGrammar} and \ref{sec:LTLFutureSemantics} with the past LTL operators; once, previous and has-always-been; defined in \ref{sec:LTLPastGrammar} and \ref{sec:LTLPastSemantics}.\\
\\
\noindent
We make the conjecture that any security property we wish to formulate, can be written in terms of past events and past operators.  For example, a policy might be that there should be no audio taken from the microphone if there are no calls in progress.  In that case we would formulate a property that is satisfied when an audio read event \textit{has} taken place, and there were no \textit{preceding} call events.  Such a property is concerned with events occurring in the past.  We assert that all security properties can be written in this manner.  We support this conjecture by calling upon Pneuli's description of the past temporal operators.  He describes the past operators as `a symmetric counterpart to each of the future operators.  While the future formula describes a property holding at a suffix of the model (...), a past formula describes a property of a prefix of the model'\cite{PnueliFuturePastOperators}.  Indeed, the difference between the standard algorithm and the reverse algorithm is that the semantics of the future operators operate upon the trace suffix and the past operators upon the prefix.\\
\\
\noindent
The upshot is that the collusion property we formulated in chapter \ref{chap:Monitoring For Security Properties} as $\LTLeventually(q \land \LTLeventually(s \land \LTLeventually(r \land \LTLeventually p)))$, must be altered to define the property in terms of past events rather than future.  In future parlance the property reads `eventually there is a query and eventually there is a send and eventually there is a receive and eventually there is a publish'.  We reformulate this to read `once there was a publish and once there was a receive and once there was a send and once there was a query'.  Written as LTL the eventually operator, $\LTLeventually$, is swapped for the once operator, $\LTLonce$, and the order of the events is reversed.  The formula then appears:\\

$$\LTLonce(p \land \LTLonce(r \land \LTLonce(s \land \LTLonce q)))$$\\  

\begin{myEx} Collusion Formula Examples\\

Examples of the collusion formula evaluated over a positive and negative trace are in appendix \ref{app:PastLogicCollusionFormulaExamples}.

\qed
\end{myEx}

\newpage
\section{Evaluation Using the \RRH\ Algorithm}
\label{sec:EvaluationUsingRRH}

The \RRH\ algorithm differs from the standard algorithm only in the order that the trace events are evaluated and the operators that comprise the formula.  Otherwise it follows the same two phases as the standard algorithm:

\begin{enumerate}
\item Initialisation
\item Evaluation
\end{enumerate}

\subsection{Phase 1 - Initialisation}
\label{sec:Initialisation}

The initialisation phase of the \RRH\ algorithm is identical to that of the standard algorithm, described in Section \ref{sec:InitialisationPhase}, except that a register called \textit{previous} is produced instead of a register called \textit{next}.  A brief summary of the steps are:

\begin{enumerate}
\item Produce a syntax tree from the formula.
\item Traverse the tree in breadth-first order to produce a list of subformulae.
\item Construct two registers from arrays, where the length of each array is equal to the length of the subformula list.  The registers are named \textit{now} and \textit{previous}
\item Traverse the subformula list and both registers simultaneously, from the first to the last element.  Associate the array elements with the subformula.
\item Assign truth values to the elements in the \textit{previous} register, according to the semantics over an empty trace, of the outermost operator in the associated subformula.
\end{enumerate}

The following example demonstrates the process:

\begin{myEx} Initialising the \RRH\ Algorithm\\

The formula $\varphi = \LTLalwaysbeen((r \,S q) \rightarrow \LTLonce(q \rightarrow \LTLprevious p))$ produces subformulae:

\begin{flushleft}
$ \varphi_{1} = \LTLalwaysbeen((r \,S \,q) \rightarrow \LTLonce(q \rightarrow \LTLprevious p)) $ \\
$ \varphi_{2} = ((r \,S \,q) \rightarrow \LTLonce(q \rightarrow \LTLprevious r)) $ \\
$ \varphi_{3} = (r \,S \,q) $ \\
$ \varphi_{4} = \LTLonce(q \rightarrow \LTLprevious p) $ \\
$ \varphi_{5} = r $ \\
$ \varphi_{6} = q $ \\
$ \varphi_{7} = (q \rightarrow \LTLprevious p) $ \\
$ \varphi_{8} = q $ \\
$ \varphi_{9} = \LTLprevious p $ \\
$ \varphi_{10} = p $ 
\end{flushleft}
\newpage
When arranged in breadth-first order as arrays, they appear:

\noindent
\begin{tabular}{rr|c|c|c|c|c|c|c|c|c|c|} &
\multicolumn{1}{c}{} &
\multicolumn{1}{c} {$ \varphi_{1}$} &
\multicolumn{1}{c} {$ \varphi_{2}$} &
\multicolumn{1}{c} {$ \varphi_{3}$} &
\multicolumn{1}{c} {$ \varphi_{4}$} &
\multicolumn{1}{c} {$ \varphi_{5}$} &
\multicolumn{1}{c} {$ \varphi_{6}$} &
\multicolumn{1}{c} {$ \varphi_{7}$} &
\multicolumn{1}{c} {$ \varphi_{8}$} & 
\multicolumn{1}{c} {$ \varphi_{9}$} & 
\multicolumn{1}{c} {$ \varphi_{10}$} \\
\cline{3-12}
& previous & $\LTLalwaysbeen$ & $\rightarrow$ & $S$ & $\LTLonce$ & $r$ & $q$ & $\rightarrow$ & $q$ & $\LTLprevious$ & $p$ \\
\cline{3-12}
\\
\multicolumn{1}{c}{} &
\multicolumn{1}{c}{} &
\multicolumn{1}{c} {$ \varphi_{1}$} &
\multicolumn{1}{c} {$ \varphi_{2}$} &
\multicolumn{1}{c} {$ \varphi_{3}$} &
\multicolumn{1}{c} {$ \varphi_{4}$} &
\multicolumn{1}{c} {$ \varphi_{5}$} &
\multicolumn{1}{c} {$ \varphi_{6}$} &
\multicolumn{1}{c} {$ \varphi_{7}$} &
\multicolumn{1}{c} {$ \varphi_{8}$} & 
\multicolumn{1}{c} {$ \varphi_{9}$} & 
\multicolumn{1}{c} {$ \varphi_{10}$} \\
\cline{3-12}
& now & $\LTLalwaysbeen$ & $\rightarrow$ & $S$ & $\LTLonce$ & $r$ & $q$ & $\rightarrow$ & $q$ & $\LTLprevious$ & $p$ \\
\cline{3-12}
\end{tabular}\\
\\
\\
The elements of the \textit{previous} register are then assigned truth values according to the past operator semantics over the empty trace, defined in \ref{def:Past Empty trace semantics}, for the operator at the root of the corresponding subformula:

\noindent
\begin{tabular}{rr|c|c|c|c|c|c|c|c|c|c|} &
\multicolumn{1}{c}{} &
\multicolumn{1}{c} {$ \varphi_{1}$} &
\multicolumn{1}{c} {$ \varphi_{2}$} &
\multicolumn{1}{c} {$ \varphi_{3}$} &
\multicolumn{1}{c} {$ \varphi_{4}$} &
\multicolumn{1}{c} {$ \varphi_{5}$} &
\multicolumn{1}{c} {$ \varphi_{6}$} &
\multicolumn{1}{c} {$ \varphi_{7}$} &
\multicolumn{1}{c} {$ \varphi_{8}$} & 
\multicolumn{1}{c} {$ \varphi_{9}$} & 
\multicolumn{1}{c} {$ \varphi_{10}$} \\
\cline{3-12}
& previous & $ \top $ & $ \top $ & $ \bot $ & $ \top $ & $ \bot $ & $ \bot $ & $ \top $ & $ \bot $ & $ \bot $ & $ \bot $ \\
\cline{3-12}
\end{tabular}\\

\qed
\end{myEx}

\subsection{Phase 2 - Evaluation}
\label{sec:Evaluation}

The evaluation phase determines if a trace satisfies the formula.  The formula is said to be satisfied if it evaluates to $\top$ after considering every event in the trace, from the first event to the last.

\begin{description}
\item[Step 1] Take an event from the trace in the order of first event to last.  With that event, iterate over the \textit{now} register from last element downto the first and evaluate the corresponding subformula over the trace event.  Evaluate according to the semantics from Definition \ref{def:PastNon-emptyTraceSemantics} for the root operator of the subformula.  Assign the resulting truth value to the element.

\item[Step 2] \textit{previous[i]} = \textit{now[i]} for all i ranging from 1..length(\textit{now}).

\item[Step 3] Repeat steps 1 and 2 with the succeeding trace event until there is no succeeding events.

\item[Step 4] The final result of evaluating the formula over the trace is found as the truth value of the first element in the \textit{previous} register because this element corresponds to the root of the subformula tree.
\end{description}

In a similar fashion to the standard algorithm, the semantics of the past operators in Definition \ref{def:PastNon-emptyTraceSemantics} must be adapted for use in the \RRH\ algorithm.  It is a case of replacing terms from each operator's semantic statement with references to their algorithmic counterparts.  Subformulae of $\varphi$ are replaced with elements within the \textit{now} and \textit{previous} registers by the use of indices $i$, $j$ and $k$.  The index $i$ maps to the element corresponding to the subformulas operator.  Indices $j$ and $k$ map to the elements corresponding to the left and right operands respectively.\\
\newpage
\begin{definition}Algorithmic Past Operator Semantics
\label{def:AlgorithmicPastOperatorSemantics}
\\\\
For a given trace event $e$, for all $i$ ranging from length(\textit{now})..1, if the $i^{th}$ outermost operator is:

\indent
\begin{enumerate}[start = 10]
\item literal l, then \textit{now}[$i$] $\leftarrow$ (l == $e$)
\item $ \neg $ then \textit{now}[$i$] $ \leftarrow $ NOT(\textit{now}[$j$]).
\item $ \lor $ then \textit{now}[$i$] $ \leftarrow $ \textit{now}[$j$] OR \textit{now}[$k$]. 
\item $ \land $ then \textit{now}[$i$] $ \leftarrow $ \textit{now}[$j$] AND \textit{now}[$k$]. 
\item $ \rightarrow $ then \textit{now}[$i$] $ \leftarrow $ NOT(\textit{now}[$j$]) OR \textit{now}[$k$]. 
\item $ \LTLprevious $ then \textit{now}[$i$] $ \leftarrow $ \textit{previous}[$j$].
\item $ \LTLalwaysbeen $ then \textit{now}[$i$] $ \leftarrow $ \textit{now}[$j$] AND \textit{previous}[$i$].
\item $ \LTLonce $ then \textit{now}[$i$] $ \leftarrow $ \textit{now}[$j$] OR \textit{previous}[$i$].
\item S then \textit{now}[$i$] $ \leftarrow $ \textit{now}[$k$] OR (\textit{now}[$j$] AND \textit{previous}[$i$]).
\\
\end{enumerate}
\end{definition}

The example below runs through the evaluation of a formula over a short trace:

\begin{myEx} Using \RRH\ to evaluate a formula\\

$t = \langle p, q, r \rangle$\\
\indent $\varphi = \LTLalwaysbeen((r \,S\ q) \rightarrow \LTLonce(q \rightarrow \LTLprevious p))$
\\
% =====
% Event 1
% =====
\textbf{\item Evaluation Phase - Step 1}\\

\subitem \underline{Event 1, Iteration 1: $t = \langle p, q, r \rangle, e = p, i = 10$}\\
\\
Registers before:

\begin{tabular}{rr|c|c|c|c|c|c|c|c|c|c|} &
\multicolumn{1}{c}{} &
\multicolumn{1}{c} {$ \varphi_{1}$} &
\multicolumn{1}{c} {$ \varphi_{2}$} &
\multicolumn{1}{c} {$ \varphi_{3}$} &
\multicolumn{1}{c} {$ \varphi_{4}$} &
\multicolumn{1}{c} {$ \varphi_{5}$} &
\multicolumn{1}{c} {$ \varphi_{6}$} &
\multicolumn{1}{c} {$ \varphi_{7}$} &
\multicolumn{1}{c} {$ \varphi_{8}$} & 
\multicolumn{1}{c} {$ \varphi_{9}$} & 
\multicolumn{1}{c} {$ \varphi_{10}$} \\
\cline{3-12}
& previous & $ \top $ & $ \top $ & $ \bot $ & $ \top $ & $ \bot $ & $ \bot $ & $ \top $ & $ \bot $ & $ \bot $ & $ \bot $ \\
\cline{3-12}
\\
\multicolumn{1}{c}{} &
\multicolumn{1}{c}{} &
\multicolumn{1}{c} {$ \varphi_{1}$} &
\multicolumn{1}{c} {$ \varphi_{2}$} &
\multicolumn{1}{c} {$ \varphi_{3}$} &
\multicolumn{1}{c} {$ \varphi_{4}$} &
\multicolumn{1}{c} {$ \varphi_{5}$} &
\multicolumn{1}{c} {$ \varphi_{6}$} &
\multicolumn{1}{c} {$ \varphi_{7}$} &
\multicolumn{1}{c} {$ \varphi_{8}$} & 
\multicolumn{1}{c} {$ \varphi_{9}$} & 
\multicolumn{1}{c} {$ \varphi_{10}$} \\
\cline{3-12}
& now & $\LTLalwaysbeen$ & $\rightarrow$ & $S$ & $\LTLonce$ & $r$ & $q$ & $\rightarrow$ & $q$ & $\LTLprevious$ & $p$ \\
\cline{3-12}
\end{tabular}\\
\\
\\
The \RRH\ algorithm evaluates the earliest trace event first rather than last, in this case the earliest event is $p$.  Each event is evaluated over the formula $\varphi$ in the same way as the standard algorithm; the satisfaction of each subformula from the deepest to the root is evaluated is evaluated by iterating through the elements of the \textit{now} register and applying the semantic rules from Definition \ref{def:AlgorithmicPastOperatorSemantics} to the subformula corresponding to the element indexed by $i$.  To first element to be evaluated is $now[10]$, and the subformula is the literal $p$.  According to rule 10 of  Definition \ref{def:AlgorithmicPastOperatorSemantics}, $now[10]$ takes the value $\top$ because the event equals the literal.\\
\\
Registers after:

\begin{tabular}{rr|c|c|c|c|c|c|c|c|c|c|} &
\multicolumn{1}{c}{} &
\multicolumn{1}{c} {$ \varphi_{1}$} &
\multicolumn{1}{c} {$ \varphi_{2}$} &
\multicolumn{1}{c} {$ \varphi_{3}$} &
\multicolumn{1}{c} {$ \varphi_{4}$} &
\multicolumn{1}{c} {$ \varphi_{5}$} &
\multicolumn{1}{c} {$ \varphi_{6}$} &
\multicolumn{1}{c} {$ \varphi_{7}$} &
\multicolumn{1}{c} {$ \varphi_{8}$} & 
\multicolumn{1}{c} {$ \varphi_{9}$} & 
\multicolumn{1}{c} {$ \varphi_{10}$} \\
\cline{3-12}
& previous & $ \top $ & $ \top $ & $ \bot $ & $ \top $ & $ \bot $ & $ \bot $ & $ \top $ & $ \bot $ & $ \bot $ & $ \bot $ \\
\cline{3-12}
\\
\multicolumn{1}{c}{} &
\multicolumn{1}{c}{} &
\multicolumn{1}{c} {$ \varphi_{1}$} &
\multicolumn{1}{c} {$ \varphi_{2}$} &
\multicolumn{1}{c} {$ \varphi_{3}$} &
\multicolumn{1}{c} {$ \varphi_{4}$} &
\multicolumn{1}{c} {$ \varphi_{5}$} &
\multicolumn{1}{c} {$ \varphi_{6}$} &
\multicolumn{1}{c} {$ \varphi_{7}$} &
\multicolumn{1}{c} {$ \varphi_{8}$} & 
\multicolumn{1}{c} {$ \varphi_{9}$} & 
\multicolumn{1}{c} {$ \varphi_{10}$} \\
\cline{3-12}
& now & $\LTLalwaysbeen$ & $\rightarrow$ & $S$ & $\LTLonce$ & $r$ & $q$ & $\rightarrow$ & $q$ & $\LTLprevious$ & $\top$ \\
\cline{3-12}
\end{tabular}\\
\\
\\
\subitem \underline{Event 1, Iteration 2: $t = \langle p, q, r \rangle, e = p, i = 9, j = 10$}\\
\\
Registers before:

\begin{tabular}{rr|c|c|c|c|c|c|c|c|c|c|} &
\multicolumn{1}{c}{} &
\multicolumn{1}{c} {$ \varphi_{1}$} &
\multicolumn{1}{c} {$ \varphi_{2}$} &
\multicolumn{1}{c} {$ \varphi_{3}$} &
\multicolumn{1}{c} {$ \varphi_{4}$} &
\multicolumn{1}{c} {$ \varphi_{5}$} &
\multicolumn{1}{c} {$ \varphi_{6}$} &
\multicolumn{1}{c} {$ \varphi_{7}$} &
\multicolumn{1}{c} {$ \varphi_{8}$} & 
\multicolumn{1}{c} {$ \varphi_{9}$} & 
\multicolumn{1}{c} {$ \varphi_{10}$} \\
\cline{3-12}
& previous & $ \top $ & $ \top $ & $ \bot $ & $ \top $ & $ \bot $ & $ \bot $ & $ \top $ & $ \bot $ & $ \bot $ & $ \bot $ \\
\cline{3-12}
\\
\multicolumn{1}{c}{} &
\multicolumn{1}{c}{} &
\multicolumn{1}{c} {$ \varphi_{1}$} &
\multicolumn{1}{c} {$ \varphi_{2}$} &
\multicolumn{1}{c} {$ \varphi_{3}$} &
\multicolumn{1}{c} {$ \varphi_{4}$} &
\multicolumn{1}{c} {$ \varphi_{5}$} &
\multicolumn{1}{c} {$ \varphi_{6}$} &
\multicolumn{1}{c} {$ \varphi_{7}$} &
\multicolumn{1}{c} {$ \varphi_{8}$} & 
\multicolumn{1}{c} {$ \varphi_{9}$} & 
\multicolumn{1}{c} {$ \varphi_{10}$} \\
\cline{3-12}
& now & $\LTLalwaysbeen$ & $\rightarrow$ & $S$ & $\LTLonce$ & $r$ & $q$ & $\rightarrow$ & $q$ & $\LTLprevious$ & $\top$ \\
\cline{3-12}
\end{tabular}\\
\\
\\
The second iteration evaluates event $p$ over $now[9]$.  Rule 15 applies because operator is the previous operator.  The left operand is indexed by $j$, which has the value 10, therefore $now[9]$ is assigned the value $\bot$ from the $previous[10]$ element according to rule 15.\\
\\
Registers after:

\begin{tabular}{rr|c|c|c|c|c|c|c|c|c|c|} &
\multicolumn{1}{c}{} &
\multicolumn{1}{c} {$ \varphi_{1}$} &
\multicolumn{1}{c} {$ \varphi_{2}$} &
\multicolumn{1}{c} {$ \varphi_{3}$} &
\multicolumn{1}{c} {$ \varphi_{4}$} &
\multicolumn{1}{c} {$ \varphi_{5}$} &
\multicolumn{1}{c} {$ \varphi_{6}$} &
\multicolumn{1}{c} {$ \varphi_{7}$} &
\multicolumn{1}{c} {$ \varphi_{8}$} & 
\multicolumn{1}{c} {$ \varphi_{9}$} & 
\multicolumn{1}{c} {$ \varphi_{10}$} \\
\cline{3-12}
& previous & $ \top $ & $ \top $ & $ \bot $ & $ \top $ & $ \bot $ & $ \bot $ & $ \top $ & $ \bot $ & $ \bot $ & $ \bot $ \\
\cline{3-12}
\\
\multicolumn{1}{c}{} &
\multicolumn{1}{c}{} &
\multicolumn{1}{c} {$ \varphi_{1}$} &
\multicolumn{1}{c} {$ \varphi_{2}$} &
\multicolumn{1}{c} {$ \varphi_{3}$} &
\multicolumn{1}{c} {$ \varphi_{4}$} &
\multicolumn{1}{c} {$ \varphi_{5}$} &
\multicolumn{1}{c} {$ \varphi_{6}$} &
\multicolumn{1}{c} {$ \varphi_{7}$} &
\multicolumn{1}{c} {$ \varphi_{8}$} & 
\multicolumn{1}{c} {$ \varphi_{9}$} & 
\multicolumn{1}{c} {$ \varphi_{10}$} \\
\cline{3-12}
& now & $\LTLalwaysbeen$ & $\rightarrow$ & $S$ & $\LTLonce$ & $r$ & $q$ & $\rightarrow$ & $q$ & $\bot$ & $\top$ \\
\cline{3-12}
\end{tabular}\\
\\
\\
\newpage
\subitem \underline{Event 1, Iteration 3: $t = \langle p, q, r \rangle, e = p, i = 8$}\\
\\
Registers before:

\begin{tabular}{rr|c|c|c|c|c|c|c|c|c|c|} &
\multicolumn{1}{c}{} &
\multicolumn{1}{c} {$ \varphi_{1}$} &
\multicolumn{1}{c} {$ \varphi_{2}$} &
\multicolumn{1}{c} {$ \varphi_{3}$} &
\multicolumn{1}{c} {$ \varphi_{4}$} &
\multicolumn{1}{c} {$ \varphi_{5}$} &
\multicolumn{1}{c} {$ \varphi_{6}$} &
\multicolumn{1}{c} {$ \varphi_{7}$} &
\multicolumn{1}{c} {$ \varphi_{8}$} & 
\multicolumn{1}{c} {$ \varphi_{9}$} & 
\multicolumn{1}{c} {$ \varphi_{10}$} \\
\cline{3-12}
& previous & $ \top $ & $ \top $ & $ \bot $ & $ \top $ & $ \bot $ & $ \bot $ & $ \top $ & $ \bot $ & $ \bot $ & $ \bot $ \\
\cline{3-12}
\\
\multicolumn{1}{c}{} &
\multicolumn{1}{c}{} &
\multicolumn{1}{c} {$ \varphi_{1}$} &
\multicolumn{1}{c} {$ \varphi_{2}$} &
\multicolumn{1}{c} {$ \varphi_{3}$} &
\multicolumn{1}{c} {$ \varphi_{4}$} &
\multicolumn{1}{c} {$ \varphi_{5}$} &
\multicolumn{1}{c} {$ \varphi_{6}$} &
\multicolumn{1}{c} {$ \varphi_{7}$} &
\multicolumn{1}{c} {$ \varphi_{8}$} & 
\multicolumn{1}{c} {$ \varphi_{9}$} & 
\multicolumn{1}{c} {$ \varphi_{10}$} \\
\cline{3-12}
& now & $\LTLalwaysbeen$ & $\rightarrow$ & $S$ & $\LTLonce$ & $r$ & $q$ & $\rightarrow$ & $q$ & $\bot$ & $\top$ \\
\cline{3-12}
\end{tabular}\\
\\
\\
The third iteration evaluates the $now[8]$ element which is another literal subformula.  This time the element is assigned $\bot$ according to rule 10 because the subformula is $q$ and does not match the event $p$.\\
\\
Registers after:

\begin{tabular}{rr|c|c|c|c|c|c|c|c|c|c|} &
\multicolumn{1}{c}{} &
\multicolumn{1}{c} {$ \varphi_{1}$} &
\multicolumn{1}{c} {$ \varphi_{2}$} &
\multicolumn{1}{c} {$ \varphi_{3}$} &
\multicolumn{1}{c} {$ \varphi_{4}$} &
\multicolumn{1}{c} {$ \varphi_{5}$} &
\multicolumn{1}{c} {$ \varphi_{6}$} &
\multicolumn{1}{c} {$ \varphi_{7}$} &
\multicolumn{1}{c} {$ \varphi_{8}$} & 
\multicolumn{1}{c} {$ \varphi_{9}$} & 
\multicolumn{1}{c} {$ \varphi_{10}$} \\
\cline{3-12}
& previous & $ \top $ & $ \top $ & $ \bot $ & $ \top $ & $ \bot $ & $ \bot $ & $ \top $ & $ \bot $ & $ \bot $ & $ \bot $ \\
\cline{3-12}
\\
\multicolumn{1}{c}{} &
\multicolumn{1}{c}{} &
\multicolumn{1}{c} {$ \varphi_{1}$} &
\multicolumn{1}{c} {$ \varphi_{2}$} &
\multicolumn{1}{c} {$ \varphi_{3}$} &
\multicolumn{1}{c} {$ \varphi_{4}$} &
\multicolumn{1}{c} {$ \varphi_{5}$} &
\multicolumn{1}{c} {$ \varphi_{6}$} &
\multicolumn{1}{c} {$ \varphi_{7}$} &
\multicolumn{1}{c} {$ \varphi_{8}$} & 
\multicolumn{1}{c} {$ \varphi_{9}$} & 
\multicolumn{1}{c} {$ \varphi_{10}$} \\
\cline{3-12}
& now & $\LTLalwaysbeen$ & $\rightarrow$ & $S$ & $\LTLonce$ & $r$ & $q$ & $\rightarrow$ & $\bot$ & $\bot$ & $\top$ \\
\cline{3-12}
\end{tabular}\\
\\
\\
\subitem \underline{Event 1, Iteration 4: $t = \langle p, q, r \rangle, e = p, i = 7, j = 8, k = 9$}\\
\\
Registers before:

\begin{tabular}{rr|c|c|c|c|c|c|c|c|c|c|} &
\multicolumn{1}{c}{} &
\multicolumn{1}{c} {$ \varphi_{1}$} &
\multicolumn{1}{c} {$ \varphi_{2}$} &
\multicolumn{1}{c} {$ \varphi_{3}$} &
\multicolumn{1}{c} {$ \varphi_{4}$} &
\multicolumn{1}{c} {$ \varphi_{5}$} &
\multicolumn{1}{c} {$ \varphi_{6}$} &
\multicolumn{1}{c} {$ \varphi_{7}$} &
\multicolumn{1}{c} {$ \varphi_{8}$} & 
\multicolumn{1}{c} {$ \varphi_{9}$} & 
\multicolumn{1}{c} {$ \varphi_{10}$} \\
\cline{3-12}
& previous & $ \top $ & $ \top $ & $ \bot $ & $ \top $ & $ \bot $ & $ \bot $ & $ \top $ & $ \bot $ & $ \bot $ & $ \bot $ \\
\cline{3-12}
\\
\multicolumn{1}{c}{} &
\multicolumn{1}{c}{} &
\multicolumn{1}{c} {$ \varphi_{1}$} &
\multicolumn{1}{c} {$ \varphi_{2}$} &
\multicolumn{1}{c} {$ \varphi_{3}$} &
\multicolumn{1}{c} {$ \varphi_{4}$} &
\multicolumn{1}{c} {$ \varphi_{5}$} &
\multicolumn{1}{c} {$ \varphi_{6}$} &
\multicolumn{1}{c} {$ \varphi_{7}$} &
\multicolumn{1}{c} {$ \varphi_{8}$} & 
\multicolumn{1}{c} {$ \varphi_{9}$} & 
\multicolumn{1}{c} {$ \varphi_{10}$} \\
\cline{3-12}
& now & $\LTLalwaysbeen$ & $\rightarrow$ & $S$ & $\LTLonce$ & $r$ & $q$ & $\rightarrow$ & $\bot$ & $\bot$ & $\top$ \\
\cline{3-12}
\end{tabular}\\
\\
\\
Element $now[7]$ is an implies formula and is assigned $\top$ by rule 14 because the left operand $now[8]$ is $\bot$.\\
\\
\newpage
Registers after:

\begin{tabular}{rr|c|c|c|c|c|c|c|c|c|c|} &
\multicolumn{1}{c}{} &
\multicolumn{1}{c} {$ \varphi_{1}$} &
\multicolumn{1}{c} {$ \varphi_{2}$} &
\multicolumn{1}{c} {$ \varphi_{3}$} &
\multicolumn{1}{c} {$ \varphi_{4}$} &
\multicolumn{1}{c} {$ \varphi_{5}$} &
\multicolumn{1}{c} {$ \varphi_{6}$} &
\multicolumn{1}{c} {$ \varphi_{7}$} &
\multicolumn{1}{c} {$ \varphi_{8}$} & 
\multicolumn{1}{c} {$ \varphi_{9}$} & 
\multicolumn{1}{c} {$ \varphi_{10}$} \\
\cline{3-12}
& previous & $ \top $ & $ \top $ & $ \bot $ & $ \top $ & $ \bot $ & $ \bot $ & $ \top $ & $ \bot $ & $ \bot $ & $ \bot $ \\
\cline{3-12}
\\
\multicolumn{1}{c}{} &
\multicolumn{1}{c}{} &
\multicolumn{1}{c} {$ \varphi_{1}$} &
\multicolumn{1}{c} {$ \varphi_{2}$} &
\multicolumn{1}{c} {$ \varphi_{3}$} &
\multicolumn{1}{c} {$ \varphi_{4}$} &
\multicolumn{1}{c} {$ \varphi_{5}$} &
\multicolumn{1}{c} {$ \varphi_{6}$} &
\multicolumn{1}{c} {$ \varphi_{7}$} &
\multicolumn{1}{c} {$ \varphi_{8}$} & 
\multicolumn{1}{c} {$ \varphi_{9}$} & 
\multicolumn{1}{c} {$ \varphi_{10}$} \\
\cline{3-12}
& now & $\LTLalwaysbeen$ & $\rightarrow$ & $S$ & $\LTLonce$ & $r$ & $q$ & $\top$ & $\bot$ & $\bot$ & $\top$ \\
\cline{3-12}
\end{tabular}\\
\\
\\
\subitem \underline{Event 1, Iteration 5: $t = \langle p, q, r \rangle, e = p, i = 6$}\\
\\
Registers before:

\begin{tabular}{rr|c|c|c|c|c|c|c|c|c|c|} &
\multicolumn{1}{c}{} &
\multicolumn{1}{c} {$ \varphi_{1}$} &
\multicolumn{1}{c} {$ \varphi_{2}$} &
\multicolumn{1}{c} {$ \varphi_{3}$} &
\multicolumn{1}{c} {$ \varphi_{4}$} &
\multicolumn{1}{c} {$ \varphi_{5}$} &
\multicolumn{1}{c} {$ \varphi_{6}$} &
\multicolumn{1}{c} {$ \varphi_{7}$} &
\multicolumn{1}{c} {$ \varphi_{8}$} & 
\multicolumn{1}{c} {$ \varphi_{9}$} & 
\multicolumn{1}{c} {$ \varphi_{10}$} \\
\cline{3-12}
& previous & $ \top $ & $ \top $ & $ \bot $ & $ \top $ & $ \bot $ & $ \bot $ & $ \top $ & $ \bot $ & $ \bot $ & $ \bot $ \\
\cline{3-12}
\\
\multicolumn{1}{c}{} &
\multicolumn{1}{c}{} &
\multicolumn{1}{c} {$ \varphi_{1}$} &
\multicolumn{1}{c} {$ \varphi_{2}$} &
\multicolumn{1}{c} {$ \varphi_{3}$} &
\multicolumn{1}{c} {$ \varphi_{4}$} &
\multicolumn{1}{c} {$ \varphi_{5}$} &
\multicolumn{1}{c} {$ \varphi_{6}$} &
\multicolumn{1}{c} {$ \varphi_{7}$} &
\multicolumn{1}{c} {$ \varphi_{8}$} & 
\multicolumn{1}{c} {$ \varphi_{9}$} & 
\multicolumn{1}{c} {$ \varphi_{10}$} \\
\cline{3-12}
& now & $\LTLalwaysbeen$ & $\rightarrow$ & $S$ & $\LTLonce$ & $r$ & $q$ & $\top$ & $\bot$ & $\bot$ & $\top$ \\
\cline{3-12}
\end{tabular}\\
\\
\\
Element $now[6]$ is another literal formula and is assigned $\bot$ according to rule 10 because the event $p$ does not match the subformula $q$. \\
\\
Registers after:

\begin{tabular}{rr|c|c|c|c|c|c|c|c|c|c|} &
\multicolumn{1}{c}{} &
\multicolumn{1}{c} {$ \varphi_{1}$} &
\multicolumn{1}{c} {$ \varphi_{2}$} &
\multicolumn{1}{c} {$ \varphi_{3}$} &
\multicolumn{1}{c} {$ \varphi_{4}$} &
\multicolumn{1}{c} {$ \varphi_{5}$} &
\multicolumn{1}{c} {$ \varphi_{6}$} &
\multicolumn{1}{c} {$ \varphi_{7}$} &
\multicolumn{1}{c} {$ \varphi_{8}$} & 
\multicolumn{1}{c} {$ \varphi_{9}$} & 
\multicolumn{1}{c} {$ \varphi_{10}$} \\
\cline{3-12}
& previous & $ \top $ & $ \top $ & $ \bot $ & $ \top $ & $ \bot $ & $ \bot $ & $ \top $ & $ \bot $ & $ \bot $ & $ \bot $ \\
\cline{3-12}
\\
\multicolumn{1}{c}{} &
\multicolumn{1}{c}{} &
\multicolumn{1}{c} {$ \varphi_{1}$} &
\multicolumn{1}{c} {$ \varphi_{2}$} &
\multicolumn{1}{c} {$ \varphi_{3}$} &
\multicolumn{1}{c} {$ \varphi_{4}$} &
\multicolumn{1}{c} {$ \varphi_{5}$} &
\multicolumn{1}{c} {$ \varphi_{6}$} &
\multicolumn{1}{c} {$ \varphi_{7}$} &
\multicolumn{1}{c} {$ \varphi_{8}$} & 
\multicolumn{1}{c} {$ \varphi_{9}$} & 
\multicolumn{1}{c} {$ \varphi_{10}$} \\
\cline{3-12}
& now & $\LTLalwaysbeen$ & $\rightarrow$ & $S$ & $\LTLonce$ & $r$ & $\bot$ & $\top$ & $\bot$ & $\bot$ & $\top$ \\
\cline{3-12}
\end{tabular}\\
\\
\\
\subitem \underline{Event 1, Iteration 6: $t = \langle p, q, r \rangle, e = p, i = 5$}\\
\\
Registers before:

\begin{tabular}{rr|c|c|c|c|c|c|c|c|c|c|} &
\multicolumn{1}{c}{} &
\multicolumn{1}{c} {$ \varphi_{1}$} &
\multicolumn{1}{c} {$ \varphi_{2}$} &
\multicolumn{1}{c} {$ \varphi_{3}$} &
\multicolumn{1}{c} {$ \varphi_{4}$} &
\multicolumn{1}{c} {$ \varphi_{5}$} &
\multicolumn{1}{c} {$ \varphi_{6}$} &
\multicolumn{1}{c} {$ \varphi_{7}$} &
\multicolumn{1}{c} {$ \varphi_{8}$} & 
\multicolumn{1}{c} {$ \varphi_{9}$} & 
\multicolumn{1}{c} {$ \varphi_{10}$} \\
\cline{3-12}
& previous & $ \top $ & $ \top $ & $ \bot $ & $ \top $ & $ \bot $ & $ \bot $ & $ \top $ & $ \bot $ & $ \bot $ & $ \bot $ \\
\cline{3-12}
\\
\multicolumn{1}{c}{} &
\multicolumn{1}{c}{} &
\multicolumn{1}{c} {$ \varphi_{1}$} &
\multicolumn{1}{c} {$ \varphi_{2}$} &
\multicolumn{1}{c} {$ \varphi_{3}$} &
\multicolumn{1}{c} {$ \varphi_{4}$} &
\multicolumn{1}{c} {$ \varphi_{5}$} &
\multicolumn{1}{c} {$ \varphi_{6}$} &
\multicolumn{1}{c} {$ \varphi_{7}$} &
\multicolumn{1}{c} {$ \varphi_{8}$} & 
\multicolumn{1}{c} {$ \varphi_{9}$} & 
\multicolumn{1}{c} {$ \varphi_{10}$} \\
\cline{3-12}
& now & $\LTLalwaysbeen$ & $\rightarrow$ & $S$ & $\LTLonce$ & $r$ & $\bot$ & $\top$ & $\bot$ & $\bot$ & $\top$ \\
\cline{3-12}
\end{tabular}\\
\\
\\
Element $now[6]$ is a literal formula $r$, it does not match the event $p$ and is assigned $\bot$ by rule 10.\\
\\
Registers after:

\begin{tabular}{rr|c|c|c|c|c|c|c|c|c|c|} &
\multicolumn{1}{c}{} &
\multicolumn{1}{c} {$ \varphi_{1}$} &
\multicolumn{1}{c} {$ \varphi_{2}$} &
\multicolumn{1}{c} {$ \varphi_{3}$} &
\multicolumn{1}{c} {$ \varphi_{4}$} &
\multicolumn{1}{c} {$ \varphi_{5}$} &
\multicolumn{1}{c} {$ \varphi_{6}$} &
\multicolumn{1}{c} {$ \varphi_{7}$} &
\multicolumn{1}{c} {$ \varphi_{8}$} & 
\multicolumn{1}{c} {$ \varphi_{9}$} & 
\multicolumn{1}{c} {$ \varphi_{10}$} \\
\cline{3-12}
& previous & $ \top $ & $ \top $ & $ \bot $ & $ \top $ & $ \bot $ & $ \bot $ & $ \top $ & $ \bot $ & $ \bot $ & $ \bot $ \\
\cline{3-12}
\\
\multicolumn{1}{c}{} &
\multicolumn{1}{c}{} &
\multicolumn{1}{c} {$ \varphi_{1}$} &
\multicolumn{1}{c} {$ \varphi_{2}$} &
\multicolumn{1}{c} {$ \varphi_{3}$} &
\multicolumn{1}{c} {$ \varphi_{4}$} &
\multicolumn{1}{c} {$ \varphi_{5}$} &
\multicolumn{1}{c} {$ \varphi_{6}$} &
\multicolumn{1}{c} {$ \varphi_{7}$} &
\multicolumn{1}{c} {$ \varphi_{8}$} & 
\multicolumn{1}{c} {$ \varphi_{9}$} & 
\multicolumn{1}{c} {$ \varphi_{10}$} \\
\cline{3-12}
& now & $\LTLalwaysbeen$ & $\rightarrow$ & $S$ & $\LTLonce$ & $\bot$ & $\bot$ & $\top$ & $\bot$ & $\bot$ & $\top$ \\
\cline{3-12}
\end{tabular}\\
\\
\\
\subitem \underline{Event 1, Iteration 7: $t = \langle p, q, r \rangle, e = p, i = 4, j = 7$}\\
\\
Registers before:

\begin{tabular}{rr|c|c|c|c|c|c|c|c|c|c|} &
\multicolumn{1}{c}{} &
\multicolumn{1}{c} {$ \varphi_{1}$} &
\multicolumn{1}{c} {$ \varphi_{2}$} &
\multicolumn{1}{c} {$ \varphi_{3}$} &
\multicolumn{1}{c} {$ \varphi_{4}$} &
\multicolumn{1}{c} {$ \varphi_{5}$} &
\multicolumn{1}{c} {$ \varphi_{6}$} &
\multicolumn{1}{c} {$ \varphi_{7}$} &
\multicolumn{1}{c} {$ \varphi_{8}$} & 
\multicolumn{1}{c} {$ \varphi_{9}$} & 
\multicolumn{1}{c} {$ \varphi_{10}$} \\
\cline{3-12}
& previous & $ \top $ & $ \top $ & $ \bot $ & $ \top $ & $ \bot $ & $ \bot $ & $ \top $ & $ \bot $ & $ \bot $ & $ \bot $ \\
\cline{3-12}
\\
\multicolumn{1}{c}{} &
\multicolumn{1}{c}{} &
\multicolumn{1}{c} {$ \varphi_{1}$} &
\multicolumn{1}{c} {$ \varphi_{2}$} &
\multicolumn{1}{c} {$ \varphi_{3}$} &
\multicolumn{1}{c} {$ \varphi_{4}$} &
\multicolumn{1}{c} {$ \varphi_{5}$} &
\multicolumn{1}{c} {$ \varphi_{6}$} &
\multicolumn{1}{c} {$ \varphi_{7}$} &
\multicolumn{1}{c} {$ \varphi_{8}$} & 
\multicolumn{1}{c} {$ \varphi_{9}$} & 
\multicolumn{1}{c} {$ \varphi_{10}$} \\
\cline{3-12}
& now & $\LTLalwaysbeen$ & $\rightarrow$ & $S$ & $\LTLonce$ & $\bot$ & $\bot$ & $\top$ & $\bot$ & $\bot$ & $\top$ \\
\cline{3-12}
\end{tabular}\\
\\
\\
Element $now[5]$ corresponds to a subformula with the once operator at its root.  Rule 17 of Definition \ref{def:AlgorithmicPastOperatorSemantics} defines the semantics of the once operator.  It is satisfied if the left operand is $\top$ now or the formula was previously satisfied.  In this case element $now[5]$ is assigned $\top$ because $previous[5]$ is $\top$.\\
\\
Registers after:

\begin{tabular}{rr|c|c|c|c|c|c|c|c|c|c|} &
\multicolumn{1}{c}{} &
\multicolumn{1}{c} {$ \varphi_{1}$} &
\multicolumn{1}{c} {$ \varphi_{2}$} &
\multicolumn{1}{c} {$ \varphi_{3}$} &
\multicolumn{1}{c} {$ \varphi_{4}$} &
\multicolumn{1}{c} {$ \varphi_{5}$} &
\multicolumn{1}{c} {$ \varphi_{6}$} &
\multicolumn{1}{c} {$ \varphi_{7}$} &
\multicolumn{1}{c} {$ \varphi_{8}$} & 
\multicolumn{1}{c} {$ \varphi_{9}$} & 
\multicolumn{1}{c} {$ \varphi_{10}$} \\
\cline{3-12}
& previous & $ \top $ & $ \top $ & $ \bot $ & $ \top $ & $ \bot $ & $ \bot $ & $ \top $ & $ \bot $ & $ \bot $ & $ \bot $ \\
\cline{3-12}
\\
\multicolumn{1}{c}{} &
\multicolumn{1}{c}{} &
\multicolumn{1}{c} {$ \varphi_{1}$} &
\multicolumn{1}{c} {$ \varphi_{2}$} &
\multicolumn{1}{c} {$ \varphi_{3}$} &
\multicolumn{1}{c} {$ \varphi_{4}$} &
\multicolumn{1}{c} {$ \varphi_{5}$} &
\multicolumn{1}{c} {$ \varphi_{6}$} &
\multicolumn{1}{c} {$ \varphi_{7}$} &
\multicolumn{1}{c} {$ \varphi_{8}$} & 
\multicolumn{1}{c} {$ \varphi_{9}$} & 
\multicolumn{1}{c} {$ \varphi_{10}$} \\
\cline{3-12}
& now & $\LTLalwaysbeen$ & $\rightarrow$ & $S$ & $\top$ & $\bot$ & $\bot$ & $\top$ & $\bot$ & $\bot$ & $\top$ \\
\cline{3-12}
\end{tabular}\\
\\
\\
\newpage
\subitem \underline{Event 1, Iteration 8: $t = \langle p, q, r \rangle, e = p, i = 3, j = 5, k = 6$}\\
\\
Registers before:

\begin{tabular}{rr|c|c|c|c|c|c|c|c|c|c|} &
\multicolumn{1}{c}{} &
\multicolumn{1}{c} {$ \varphi_{1}$} &
\multicolumn{1}{c} {$ \varphi_{2}$} &
\multicolumn{1}{c} {$ \varphi_{3}$} &
\multicolumn{1}{c} {$ \varphi_{4}$} &
\multicolumn{1}{c} {$ \varphi_{5}$} &
\multicolumn{1}{c} {$ \varphi_{6}$} &
\multicolumn{1}{c} {$ \varphi_{7}$} &
\multicolumn{1}{c} {$ \varphi_{8}$} & 
\multicolumn{1}{c} {$ \varphi_{9}$} & 
\multicolumn{1}{c} {$ \varphi_{10}$} \\
\cline{3-12}
& previous & $ \top $ & $ \top $ & $ \bot $ & $ \top $ & $ \bot $ & $ \bot $ & $ \top $ & $ \bot $ & $ \bot $ & $ \bot $ \\
\cline{3-12}
\\
\multicolumn{1}{c}{} &
\multicolumn{1}{c}{} &
\multicolumn{1}{c} {$ \varphi_{1}$} &
\multicolumn{1}{c} {$ \varphi_{2}$} &
\multicolumn{1}{c} {$ \varphi_{3}$} &
\multicolumn{1}{c} {$ \varphi_{4}$} &
\multicolumn{1}{c} {$ \varphi_{5}$} &
\multicolumn{1}{c} {$ \varphi_{6}$} &
\multicolumn{1}{c} {$ \varphi_{7}$} &
\multicolumn{1}{c} {$ \varphi_{8}$} & 
\multicolumn{1}{c} {$ \varphi_{9}$} & 
\multicolumn{1}{c} {$ \varphi_{10}$} \\
\cline{3-12}
& now & $\LTLalwaysbeen$ & $\rightarrow$ & $S$ & $\top$ & $\bot$ & $\bot$ & $\top$ & $\bot$ & $\bot$ & $\top$ \\
\cline{3-12}
\end{tabular}\\
\\
\\
The $now[3]$ element corresponds to the $(r \,S \,q)$ subformula with a since operator at its root.  The since operator evaluates $\top$ when the right operand is  $\top$ for the present event, or, the left operand is $\top$ for the present event and the subformula was $\top$ for the previous event.  The right operand is the element $now[6]$ and the left operand is the element $now[5]$.  The operator evaluates to $\bot$ because the left and right operands are both $\bot$.\\
\\
Registers after:

\begin{tabular}{rr|c|c|c|c|c|c|c|c|c|c|} &
\multicolumn{1}{c}{} &
\multicolumn{1}{c} {$ \varphi_{1}$} &
\multicolumn{1}{c} {$ \varphi_{2}$} &
\multicolumn{1}{c} {$ \varphi_{3}$} &
\multicolumn{1}{c} {$ \varphi_{4}$} &
\multicolumn{1}{c} {$ \varphi_{5}$} &
\multicolumn{1}{c} {$ \varphi_{6}$} &
\multicolumn{1}{c} {$ \varphi_{7}$} &
\multicolumn{1}{c} {$ \varphi_{8}$} & 
\multicolumn{1}{c} {$ \varphi_{9}$} & 
\multicolumn{1}{c} {$ \varphi_{10}$} \\
\cline{3-12}
& previous & $ \top $ & $ \top $ & $ \bot $ & $ \top $ & $ \bot $ & $ \bot $ & $ \top $ & $ \bot $ & $ \bot $ & $ \bot $ \\
\cline{3-12}
\\
\multicolumn{1}{c}{} &
\multicolumn{1}{c}{} &
\multicolumn{1}{c} {$ \varphi_{1}$} &
\multicolumn{1}{c} {$ \varphi_{2}$} &
\multicolumn{1}{c} {$ \varphi_{3}$} &
\multicolumn{1}{c} {$ \varphi_{4}$} &
\multicolumn{1}{c} {$ \varphi_{5}$} &
\multicolumn{1}{c} {$ \varphi_{6}$} &
\multicolumn{1}{c} {$ \varphi_{7}$} &
\multicolumn{1}{c} {$ \varphi_{8}$} & 
\multicolumn{1}{c} {$ \varphi_{9}$} & 
\multicolumn{1}{c} {$ \varphi_{10}$} \\
\cline{3-12}
& now & $\LTLalwaysbeen$ & $\rightarrow$ & $\bot$ & $\top$ & $\bot$ & $\bot$ & $\top$ & $\bot$ & $\bot$ & $\top$ \\
\cline{3-12}
\end{tabular}\\
\\
\\
\subitem \underline{Event 1, Iteration 9: $t = \langle p, q, r \rangle, e = p, i = 2, j = 3, k = 4$}\\
\\
Registers before:

\begin{tabular}{rr|c|c|c|c|c|c|c|c|c|c|} &
\multicolumn{1}{c}{} &
\multicolumn{1}{c} {$ \varphi_{1}$} &
\multicolumn{1}{c} {$ \varphi_{2}$} &
\multicolumn{1}{c} {$ \varphi_{3}$} &
\multicolumn{1}{c} {$ \varphi_{4}$} &
\multicolumn{1}{c} {$ \varphi_{5}$} &
\multicolumn{1}{c} {$ \varphi_{6}$} &
\multicolumn{1}{c} {$ \varphi_{7}$} &
\multicolumn{1}{c} {$ \varphi_{8}$} & 
\multicolumn{1}{c} {$ \varphi_{9}$} & 
\multicolumn{1}{c} {$ \varphi_{10}$} \\
\cline{3-12}
& previous & $ \top $ & $ \top $ & $ \bot $ & $ \top $ & $ \bot $ & $ \bot $ & $ \top $ & $ \bot $ & $ \bot $ & $ \bot $ \\
\cline{3-12}
\\
\multicolumn{1}{c}{} &
\multicolumn{1}{c}{} &
\multicolumn{1}{c} {$ \varphi_{1}$} &
\multicolumn{1}{c} {$ \varphi_{2}$} &
\multicolumn{1}{c} {$ \varphi_{3}$} &
\multicolumn{1}{c} {$ \varphi_{4}$} &
\multicolumn{1}{c} {$ \varphi_{5}$} &
\multicolumn{1}{c} {$ \varphi_{6}$} &
\multicolumn{1}{c} {$ \varphi_{7}$} &
\multicolumn{1}{c} {$ \varphi_{8}$} & 
\multicolumn{1}{c} {$ \varphi_{9}$} & 
\multicolumn{1}{c} {$ \varphi_{10}$} \\
\cline{3-12}
& now & $\LTLalwaysbeen$ & $\rightarrow$ & $\bot$ & $\top$ & $\bot$ & $\bot$ & $\top$ & $\bot$ & $\bot$ & $\top$ \\
\cline{3-12}
\end{tabular}\\
\\
\\
The penultimate iteration for the first event evaluates an implies subformula.  Element $now[2]$ assigned $\top$ according to rule 14 because the left operand, $now[3]$ is $\bot$.\\
\\
\newpage
Registers after:

\begin{tabular}{rr|c|c|c|c|c|c|c|c|c|c|} &
\multicolumn{1}{c}{} &
\multicolumn{1}{c} {$ \varphi_{1}$} &
\multicolumn{1}{c} {$ \varphi_{2}$} &
\multicolumn{1}{c} {$ \varphi_{3}$} &
\multicolumn{1}{c} {$ \varphi_{4}$} &
\multicolumn{1}{c} {$ \varphi_{5}$} &
\multicolumn{1}{c} {$ \varphi_{6}$} &
\multicolumn{1}{c} {$ \varphi_{7}$} &
\multicolumn{1}{c} {$ \varphi_{8}$} & 
\multicolumn{1}{c} {$ \varphi_{9}$} & 
\multicolumn{1}{c} {$ \varphi_{10}$} \\
\cline{3-12}
& previous & $ \top $ & $ \top $ & $ \bot $ & $ \top $ & $ \bot $ & $ \bot $ & $ \top $ & $ \bot $ & $ \bot $ & $ \bot $ \\
\cline{3-12}
\\
\multicolumn{1}{c}{} &
\multicolumn{1}{c}{} &
\multicolumn{1}{c} {$ \varphi_{1}$} &
\multicolumn{1}{c} {$ \varphi_{2}$} &
\multicolumn{1}{c} {$ \varphi_{3}$} &
\multicolumn{1}{c} {$ \varphi_{4}$} &
\multicolumn{1}{c} {$ \varphi_{5}$} &
\multicolumn{1}{c} {$ \varphi_{6}$} &
\multicolumn{1}{c} {$ \varphi_{7}$} &
\multicolumn{1}{c} {$ \varphi_{8}$} & 
\multicolumn{1}{c} {$ \varphi_{9}$} & 
\multicolumn{1}{c} {$ \varphi_{10}$} \\
\cline{3-12}
& now & $\LTLalwaysbeen$ & $\top$ & $\bot$ & $\top$ & $\bot$ & $\bot$ & $\top$ & $\bot$ & $\bot$ & $\top$ \\
\cline{3-12}
\end{tabular}\\
\\
\\
\subitem \underline{Event 1, Iteration 10: $t = \langle p, q, r \rangle, e = p, i = 1, j = 2$}\\
\\
Registers before:

\begin{tabular}{rr|c|c|c|c|c|c|c|c|c|c|} &
\multicolumn{1}{c}{} &
\multicolumn{1}{c} {$ \varphi_{1}$} &
\multicolumn{1}{c} {$ \varphi_{2}$} &
\multicolumn{1}{c} {$ \varphi_{3}$} &
\multicolumn{1}{c} {$ \varphi_{4}$} &
\multicolumn{1}{c} {$ \varphi_{5}$} &
\multicolumn{1}{c} {$ \varphi_{6}$} &
\multicolumn{1}{c} {$ \varphi_{7}$} &
\multicolumn{1}{c} {$ \varphi_{8}$} & 
\multicolumn{1}{c} {$ \varphi_{9}$} & 
\multicolumn{1}{c} {$ \varphi_{10}$} \\
\cline{3-12}
& previous & $ \top $ & $ \top $ & $ \bot $ & $ \top $ & $ \bot $ & $ \bot $ & $ \top $ & $ \bot $ & $ \bot $ & $ \bot $ \\
\cline{3-12}
\\
\multicolumn{1}{c}{} &
\multicolumn{1}{c}{} &
\multicolumn{1}{c} {$ \varphi_{1}$} &
\multicolumn{1}{c} {$ \varphi_{2}$} &
\multicolumn{1}{c} {$ \varphi_{3}$} &
\multicolumn{1}{c} {$ \varphi_{4}$} &
\multicolumn{1}{c} {$ \varphi_{5}$} &
\multicolumn{1}{c} {$ \varphi_{6}$} &
\multicolumn{1}{c} {$ \varphi_{7}$} &
\multicolumn{1}{c} {$ \varphi_{8}$} & 
\multicolumn{1}{c} {$ \varphi_{9}$} & 
\multicolumn{1}{c} {$ \varphi_{10}$} \\
\cline{3-12}
& now & $\LTLalwaysbeen$ & $\top$ & $\bot$ & $\top$ & $\bot$ & $\bot$ & $\top$ & $\bot$ & $\bot$ & $\top$ \\
\cline{3-12}
\end{tabular}\\
\\
\\
The final element to be evaluated is the element $now[1]$ that corresponds to the root formula and has the always been operator as its outermost operator.  The always been operator is defined by rule 16 and is $\top$ if the left operand is $\top$ for the present event and the formula is $\top$ for all previous events.  In this case it evaluates $\top$.\\
\\
Registers after:

\begin{tabular}{rr|c|c|c|c|c|c|c|c|c|c|} &
\multicolumn{1}{c}{} &
\multicolumn{1}{c} {$ \varphi_{1}$} &
\multicolumn{1}{c} {$ \varphi_{2}$} &
\multicolumn{1}{c} {$ \varphi_{3}$} &
\multicolumn{1}{c} {$ \varphi_{4}$} &
\multicolumn{1}{c} {$ \varphi_{5}$} &
\multicolumn{1}{c} {$ \varphi_{6}$} &
\multicolumn{1}{c} {$ \varphi_{7}$} &
\multicolumn{1}{c} {$ \varphi_{8}$} & 
\multicolumn{1}{c} {$ \varphi_{9}$} & 
\multicolumn{1}{c} {$ \varphi_{10}$} \\
\cline{3-12}
& previous & $ \top $ & $ \top $ & $ \bot $ & $ \top $ & $ \bot $ & $ \bot $ & $ \top $ & $ \bot $ & $ \bot $ & $ \bot $ \\
\cline{3-12}
\\
\multicolumn{1}{c}{} &
\multicolumn{1}{c}{} &
\multicolumn{1}{c} {$ \varphi_{1}$} &
\multicolumn{1}{c} {$ \varphi_{2}$} &
\multicolumn{1}{c} {$ \varphi_{3}$} &
\multicolumn{1}{c} {$ \varphi_{4}$} &
\multicolumn{1}{c} {$ \varphi_{5}$} &
\multicolumn{1}{c} {$ \varphi_{6}$} &
\multicolumn{1}{c} {$ \varphi_{7}$} &
\multicolumn{1}{c} {$ \varphi_{8}$} & 
\multicolumn{1}{c} {$ \varphi_{9}$} & 
\multicolumn{1}{c} {$ \varphi_{10}$} \\
\cline{3-12}
& now & $\top$ & $\top$ & $\bot$ & $\top$ & $\bot$ & $\bot$ & $\top$ & $\bot$ & $\bot$ & $\top$ \\
\cline{3-12}
\end{tabular}\\
\\
\\
\newpage
\textbf{\item Evaluation Phase - Step 2}\\
\\
Register before:\\

\begin{tabular}{rr|c|c|c|c|c|c|c|c|c|c|} &
\multicolumn{1}{c}{} &
\multicolumn{1}{c} {$ \varphi_{1}$} &
\multicolumn{1}{c} {$ \varphi_{2}$} &
\multicolumn{1}{c} {$ \varphi_{3}$} &
\multicolumn{1}{c} {$ \varphi_{4}$} &
\multicolumn{1}{c} {$ \varphi_{5}$} &
\multicolumn{1}{c} {$ \varphi_{6}$} &
\multicolumn{1}{c} {$ \varphi_{7}$} &
\multicolumn{1}{c} {$ \varphi_{8}$} & 
\multicolumn{1}{c} {$ \varphi_{9}$} & 
\multicolumn{1}{c} {$ \varphi_{10}$} \\
\cline{3-12}
& previous & $ \top $ & $ \top $ & $ \bot $ & $ \top $ & $ \bot $ & $ \bot $ & $ \top $ & $ \bot $ & $ \bot $ & $ \bot $ \\
\cline{3-12}
\\
\multicolumn{1}{c}{} &
\multicolumn{1}{c}{} &
\multicolumn{1}{c} {$ \varphi_{1}$} &
\multicolumn{1}{c} {$ \varphi_{2}$} &
\multicolumn{1}{c} {$ \varphi_{3}$} &
\multicolumn{1}{c} {$ \varphi_{4}$} &
\multicolumn{1}{c} {$ \varphi_{5}$} &
\multicolumn{1}{c} {$ \varphi_{6}$} &
\multicolumn{1}{c} {$ \varphi_{7}$} &
\multicolumn{1}{c} {$ \varphi_{8}$} & 
\multicolumn{1}{c} {$ \varphi_{9}$} & 
\multicolumn{1}{c} {$ \varphi_{10}$} \\
\cline{3-12}
& now & $\top$ & $\top$ & $\bot$ & $\top$ & $\bot$ & $\bot$ & $\top$ & $\bot$ & $\bot$ & $\top$ \\
\cline{3-12}
\end{tabular}\\
\\
\\
The $p$ event has been completely evaluated so step 1 of evaluation is finished.  Step 2 overwrites all the values in the \textit{previous} register with those from the \textit{now} register.\\
\\
Registers after:

\begin{tabular}{rr|c|c|c|c|c|c|c|c|c|c|} &
\multicolumn{1}{c}{} &
\multicolumn{1}{c} {$ \varphi_{1}$} &
\multicolumn{1}{c} {$ \varphi_{2}$} &
\multicolumn{1}{c} {$ \varphi_{3}$} &
\multicolumn{1}{c} {$ \varphi_{4}$} &
\multicolumn{1}{c} {$ \varphi_{5}$} &
\multicolumn{1}{c} {$ \varphi_{6}$} &
\multicolumn{1}{c} {$ \varphi_{7}$} &
\multicolumn{1}{c} {$ \varphi_{8}$} & 
\multicolumn{1}{c} {$ \varphi_{9}$} & 
\multicolumn{1}{c} {$ \varphi_{10}$} \\
\cline{3-12}
& previous & $\top$ & $\top$ & $\bot$ & $\top$ & $\bot$ & $\bot$ & $\top$ & $\bot$ & $\bot$ & $\top$ \\
\cline{3-12}
\\
\multicolumn{1}{c}{} &
\multicolumn{1}{c}{} &
\multicolumn{1}{c} {$ \varphi_{1}$} &
\multicolumn{1}{c} {$ \varphi_{2}$} &
\multicolumn{1}{c} {$ \varphi_{3}$} &
\multicolumn{1}{c} {$ \varphi_{4}$} &
\multicolumn{1}{c} {$ \varphi_{5}$} &
\multicolumn{1}{c} {$ \varphi_{6}$} &
\multicolumn{1}{c} {$ \varphi_{7}$} &
\multicolumn{1}{c} {$ \varphi_{8}$} & 
\multicolumn{1}{c} {$ \varphi_{9}$} & 
\multicolumn{1}{c} {$ \varphi_{10}$} \\
\cline{3-12}
& now & $\top$ & $\top$ & $\bot$ & $\top$ & $\bot$ & $\bot$ & $\top$ & $\bot$ & $\bot$ & $\top$ \\
\cline{3-12}
\end{tabular}\\\\
\\
\textbf{\item Evaluation Phase - Step 3}\\\\Evaluation step 3 is to repeat steps 1 and 2 for another event in the trace.  The algorithm traverses the trace from the earliest to the latest event, so the next event to be evaluated is $q$.\\
\\
\\
% =====
% Event 2
% =====
\newpage
\textbf{\item Evaluation Phase - Step 1 repeated}\\

\subitem \underline{Event 2, Iteration 1, $t = \langle p, q, r \rangle, e = q, i = 10$}\\
\\
Registers before:

\begin{tabular}{rr|c|c|c|c|c|c|c|c|c|c|} &
\multicolumn{1}{c}{} &
\multicolumn{1}{c} {$ \varphi_{1}$} &
\multicolumn{1}{c} {$ \varphi_{2}$} &
\multicolumn{1}{c} {$ \varphi_{3}$} &
\multicolumn{1}{c} {$ \varphi_{4}$} &
\multicolumn{1}{c} {$ \varphi_{5}$} &
\multicolumn{1}{c} {$ \varphi_{6}$} &
\multicolumn{1}{c} {$ \varphi_{7}$} &
\multicolumn{1}{c} {$ \varphi_{8}$} & 
\multicolumn{1}{c} {$ \varphi_{9}$} & 
\multicolumn{1}{c} {$ \varphi_{10}$} \\
\cline{3-12}
& previous & $\top$ & $\top$ & $\bot$ & $\top$ & $\bot$ & $\bot$ & $\top$ & $\bot$ & $\bot$ & $\top$ \\
\cline{3-12}
\\
\multicolumn{1}{c}{} &
\multicolumn{1}{c}{} &
\multicolumn{1}{c} {$ \varphi_{1}$} &
\multicolumn{1}{c} {$ \varphi_{2}$} &
\multicolumn{1}{c} {$ \varphi_{3}$} &
\multicolumn{1}{c} {$ \varphi_{4}$} &
\multicolumn{1}{c} {$ \varphi_{5}$} &
\multicolumn{1}{c} {$ \varphi_{6}$} &
\multicolumn{1}{c} {$ \varphi_{7}$} &
\multicolumn{1}{c} {$ \varphi_{8}$} & 
\multicolumn{1}{c} {$ \varphi_{9}$} & 
\multicolumn{1}{c} {$ \varphi_{10}$} \\
\cline{3-12}
& now & $\LTLalwaysbeen$ & $\rightarrow$ & $S$ & $\LTLonce$ & $r$ & $q$ & $\rightarrow$ & $q$ & $\LTLprevious$ & $p$ \\
\cline{3-12}
\end{tabular}\\
\\
\\
Event $q$ is evaluated over the \textit{now} register elements from $now[10]$ downto $now[1]$.  The first element to be evaluated is $now[10]$ which corresponds to the literal formula $p$.  Rule 10 from Definition \ref{def:AlgorithmicPastOperatorSemantics} applies to literal formulae and assigns $\bot$ to $now[10]$ because event $q$ does not match literal $p$.\\
\\
Registers after:

\begin{tabular}{rr|c|c|c|c|c|c|c|c|c|c|} &
\multicolumn{1}{c}{} &
\multicolumn{1}{c} {$ \varphi_{1}$} &
\multicolumn{1}{c} {$ \varphi_{2}$} &
\multicolumn{1}{c} {$ \varphi_{3}$} &
\multicolumn{1}{c} {$ \varphi_{4}$} &
\multicolumn{1}{c} {$ \varphi_{5}$} &
\multicolumn{1}{c} {$ \varphi_{6}$} &
\multicolumn{1}{c} {$ \varphi_{7}$} &
\multicolumn{1}{c} {$ \varphi_{8}$} & 
\multicolumn{1}{c} {$ \varphi_{9}$} & 
\multicolumn{1}{c} {$ \varphi_{10}$} \\
\cline{3-12}
& previous & $\top$ & $\top$ & $\bot$ & $\top$ & $\bot$ & $\bot$ & $\top$ & $\bot$ & $\bot$ & $\top$ \\
\cline{3-12}
\\
\multicolumn{1}{c}{} &
\multicolumn{1}{c}{} &
\multicolumn{1}{c} {$ \varphi_{1}$} &
\multicolumn{1}{c} {$ \varphi_{2}$} &
\multicolumn{1}{c} {$ \varphi_{3}$} &
\multicolumn{1}{c} {$ \varphi_{4}$} &
\multicolumn{1}{c} {$ \varphi_{5}$} &
\multicolumn{1}{c} {$ \varphi_{6}$} &
\multicolumn{1}{c} {$ \varphi_{7}$} &
\multicolumn{1}{c} {$ \varphi_{8}$} & 
\multicolumn{1}{c} {$ \varphi_{9}$} & 
\multicolumn{1}{c} {$ \varphi_{10}$} \\
\cline{3-12}
& now & $\LTLalwaysbeen$ & $\rightarrow$ & $S$ & $\LTLonce$ & $r$ & $q$ & $\rightarrow$ & $q$ & $\LTLprevious$ & $\bot$ \\
\cline{3-12}
\end{tabular}\\
\\
\\
\subitem \underline{Event 2, Iteration 2, $t = \langle p, q, r \rangle, e = q, i = 9, j = 10$}\\
\\
Registers before:

\begin{tabular}{rr|c|c|c|c|c|c|c|c|c|c|} &
\multicolumn{1}{c}{} &
\multicolumn{1}{c} {$ \varphi_{1}$} &
\multicolumn{1}{c} {$ \varphi_{2}$} &
\multicolumn{1}{c} {$ \varphi_{3}$} &
\multicolumn{1}{c} {$ \varphi_{4}$} &
\multicolumn{1}{c} {$ \varphi_{5}$} &
\multicolumn{1}{c} {$ \varphi_{6}$} &
\multicolumn{1}{c} {$ \varphi_{7}$} &
\multicolumn{1}{c} {$ \varphi_{8}$} & 
\multicolumn{1}{c} {$ \varphi_{9}$} & 
\multicolumn{1}{c} {$ \varphi_{10}$} \\
\cline{3-12}
& previous & $\top$ & $\top$ & $\bot$ & $\top$ & $\bot$ & $\bot$ & $\top$ & $\bot$ & $\bot$ & $\top$ \\
\cline{3-12}
\\
\multicolumn{1}{c}{} &
\multicolumn{1}{c}{} &
\multicolumn{1}{c} {$ \varphi_{1}$} &
\multicolumn{1}{c} {$ \varphi_{2}$} &
\multicolumn{1}{c} {$ \varphi_{3}$} &
\multicolumn{1}{c} {$ \varphi_{4}$} &
\multicolumn{1}{c} {$ \varphi_{5}$} &
\multicolumn{1}{c} {$ \varphi_{6}$} &
\multicolumn{1}{c} {$ \varphi_{7}$} &
\multicolumn{1}{c} {$ \varphi_{8}$} & 
\multicolumn{1}{c} {$ \varphi_{9}$} & 
\multicolumn{1}{c} {$ \varphi_{10}$} \\
\cline{3-12}
& now & $\LTLalwaysbeen$ & $\rightarrow$ & $S$ & $\LTLonce$ & $r$ & $q$ & $\rightarrow$ & $q$ & $\LTLprevious$ & $\bot$ \\
\cline{3-12}
\end{tabular}\\
\\
\\
The next element evaluated is $now[9]$, it corresponds to a formula with the previous operator at it's root, and the semantics are given by rule 15.  The element is assigned $\top$ if the left operand was satisfied by the previous event.  The \textit{previous} register tells us how the previous event was evaluated, so $now[9]$ is assigned the value from $previous[10]$, which is $\top$.\\
\\
Registers after:

\begin{tabular}{rr|c|c|c|c|c|c|c|c|c|c|} &
\multicolumn{1}{c}{} &
\multicolumn{1}{c} {$ \varphi_{1}$} &
\multicolumn{1}{c} {$ \varphi_{2}$} &
\multicolumn{1}{c} {$ \varphi_{3}$} &
\multicolumn{1}{c} {$ \varphi_{4}$} &
\multicolumn{1}{c} {$ \varphi_{5}$} &
\multicolumn{1}{c} {$ \varphi_{6}$} &
\multicolumn{1}{c} {$ \varphi_{7}$} &
\multicolumn{1}{c} {$ \varphi_{8}$} & 
\multicolumn{1}{c} {$ \varphi_{9}$} & 
\multicolumn{1}{c} {$ \varphi_{10}$} \\
\cline{3-12}
& previous & $\top$ & $\top$ & $\bot$ & $\top$ & $\bot$ & $\bot$ & $\top$ & $\bot$ & $\bot$ & $\top$ \\
\cline{3-12}
\\
\multicolumn{1}{c}{} &
\multicolumn{1}{c}{} &
\multicolumn{1}{c} {$ \varphi_{1}$} &
\multicolumn{1}{c} {$ \varphi_{2}$} &
\multicolumn{1}{c} {$ \varphi_{3}$} &
\multicolumn{1}{c} {$ \varphi_{4}$} &
\multicolumn{1}{c} {$ \varphi_{5}$} &
\multicolumn{1}{c} {$ \varphi_{6}$} &
\multicolumn{1}{c} {$ \varphi_{7}$} &
\multicolumn{1}{c} {$ \varphi_{8}$} & 
\multicolumn{1}{c} {$ \varphi_{9}$} & 
\multicolumn{1}{c} {$ \varphi_{10}$} \\
\cline{3-12}
& now & $\LTLalwaysbeen$ & $\rightarrow$ & $S$ & $\LTLonce$ & $r$ & $q$ & $\rightarrow$ & $q$ & $\top$ & $\bot$ \\
\cline{3-12}
\end{tabular}\\
\\
\\
\subitem \underline{Event 2, Iteration 3, $t = \langle p, q, r \rangle, e = q, i = 8$}\\
\\
Registers before:

\begin{tabular}{rr|c|c|c|c|c|c|c|c|c|c|} &
\multicolumn{1}{c}{} &
\multicolumn{1}{c} {$ \varphi_{1}$} &
\multicolumn{1}{c} {$ \varphi_{2}$} &
\multicolumn{1}{c} {$ \varphi_{3}$} &
\multicolumn{1}{c} {$ \varphi_{4}$} &
\multicolumn{1}{c} {$ \varphi_{5}$} &
\multicolumn{1}{c} {$ \varphi_{6}$} &
\multicolumn{1}{c} {$ \varphi_{7}$} &
\multicolumn{1}{c} {$ \varphi_{8}$} & 
\multicolumn{1}{c} {$ \varphi_{9}$} & 
\multicolumn{1}{c} {$ \varphi_{10}$} \\
\cline{3-12}
& previous & $\top$ & $\top$ & $\bot$ & $\top$ & $\bot$ & $\bot$ & $\top$ & $\bot$ & $\bot$ & $\top$ \\
\cline{3-12}
\\
\multicolumn{1}{c}{} &
\multicolumn{1}{c}{} &
\multicolumn{1}{c} {$ \varphi_{1}$} &
\multicolumn{1}{c} {$ \varphi_{2}$} &
\multicolumn{1}{c} {$ \varphi_{3}$} &
\multicolumn{1}{c} {$ \varphi_{4}$} &
\multicolumn{1}{c} {$ \varphi_{5}$} &
\multicolumn{1}{c} {$ \varphi_{6}$} &
\multicolumn{1}{c} {$ \varphi_{7}$} &
\multicolumn{1}{c} {$ \varphi_{8}$} & 
\multicolumn{1}{c} {$ \varphi_{9}$} & 
\multicolumn{1}{c} {$ \varphi_{10}$} \\
\cline{3-12}
& now & $\LTLalwaysbeen$ & $\rightarrow$ & $S$ & $\LTLonce$ & $r$ & $q$ & $\rightarrow$ & $q$ & $\top$ & $\bot$ \\
\cline{3-12}
\end{tabular}\\
\\
\\
Element $now[8]$ is a literal formula, in this case the event $q$ matches the literal formula $q$, so $now[8]$ is assigned $\top$.\\
\\
Registers after:

\begin{tabular}{rr|c|c|c|c|c|c|c|c|c|c|} &
\multicolumn{1}{c}{} &
\multicolumn{1}{c} {$ \varphi_{1}$} &
\multicolumn{1}{c} {$ \varphi_{2}$} &
\multicolumn{1}{c} {$ \varphi_{3}$} &
\multicolumn{1}{c} {$ \varphi_{4}$} &
\multicolumn{1}{c} {$ \varphi_{5}$} &
\multicolumn{1}{c} {$ \varphi_{6}$} &
\multicolumn{1}{c} {$ \varphi_{7}$} &
\multicolumn{1}{c} {$ \varphi_{8}$} & 
\multicolumn{1}{c} {$ \varphi_{9}$} & 
\multicolumn{1}{c} {$ \varphi_{10}$} \\
\cline{3-12}
& previous & $\top$ & $\top$ & $\bot$ & $\top$ & $\bot$ & $\bot$ & $\top$ & $\bot$ & $\bot$ & $\top$ \\
\cline{3-12}
\\
\multicolumn{1}{c}{} &
\multicolumn{1}{c}{} &
\multicolumn{1}{c} {$ \varphi_{1}$} &
\multicolumn{1}{c} {$ \varphi_{2}$} &
\multicolumn{1}{c} {$ \varphi_{3}$} &
\multicolumn{1}{c} {$ \varphi_{4}$} &
\multicolumn{1}{c} {$ \varphi_{5}$} &
\multicolumn{1}{c} {$ \varphi_{6}$} &
\multicolumn{1}{c} {$ \varphi_{7}$} &
\multicolumn{1}{c} {$ \varphi_{8}$} & 
\multicolumn{1}{c} {$ \varphi_{9}$} & 
\multicolumn{1}{c} {$ \varphi_{10}$} \\
\cline{3-12}
& now & $\LTLalwaysbeen$ & $\rightarrow$ & $S$ & $\LTLonce$ & $r$ & $q$ & $\rightarrow$ & $\top$ & $\top$ & $\bot$ \\
\cline{3-12}
\end{tabular}\\
\\
\\
\newpage
\subitem \underline{Event 2, Iteration 4, $t = \langle p, q, r \rangle, e = q, i = 7, j = 8, k = 9$}\\
\\
Registers before:

\begin{tabular}{rr|c|c|c|c|c|c|c|c|c|c|} &
\multicolumn{1}{c}{} &
\multicolumn{1}{c} {$ \varphi_{1}$} &
\multicolumn{1}{c} {$ \varphi_{2}$} &
\multicolumn{1}{c} {$ \varphi_{3}$} &
\multicolumn{1}{c} {$ \varphi_{4}$} &
\multicolumn{1}{c} {$ \varphi_{5}$} &
\multicolumn{1}{c} {$ \varphi_{6}$} &
\multicolumn{1}{c} {$ \varphi_{7}$} &
\multicolumn{1}{c} {$ \varphi_{8}$} & 
\multicolumn{1}{c} {$ \varphi_{9}$} & 
\multicolumn{1}{c} {$ \varphi_{10}$} \\
\cline{3-12}
& previous & $\top$ & $\top$ & $\bot$ & $\top$ & $\bot$ & $\bot$ & $\top$ & $\bot$ & $\bot$ & $\top$ \\
\cline{3-12}
\\
\multicolumn{1}{c}{} &
\multicolumn{1}{c}{} &
\multicolumn{1}{c} {$ \varphi_{1}$} &
\multicolumn{1}{c} {$ \varphi_{2}$} &
\multicolumn{1}{c} {$ \varphi_{3}$} &
\multicolumn{1}{c} {$ \varphi_{4}$} &
\multicolumn{1}{c} {$ \varphi_{5}$} &
\multicolumn{1}{c} {$ \varphi_{6}$} &
\multicolumn{1}{c} {$ \varphi_{7}$} &
\multicolumn{1}{c} {$ \varphi_{8}$} & 
\multicolumn{1}{c} {$ \varphi_{9}$} & 
\multicolumn{1}{c} {$ \varphi_{10}$} \\
\cline{3-12}
& now & $\LTLalwaysbeen$ & $\rightarrow$ & $S$ & $\LTLonce$ & $r$ & $q$ & $\rightarrow$ & $\top$ & $\top$ & $\bot$ \\
\cline{3-12}
\end{tabular}\\
\\
\\
Element $now[7]$ is an implies formula and by rule 14 is assigned $\top$ because the left operand, $now[8]$, and the right operand, $now[9]$, are both $\top$.\\
\\
Registers after:

\begin{tabular}{rr|c|c|c|c|c|c|c|c|c|c|} &
\multicolumn{1}{c}{} &
\multicolumn{1}{c} {$ \varphi_{1}$} &
\multicolumn{1}{c} {$ \varphi_{2}$} &
\multicolumn{1}{c} {$ \varphi_{3}$} &
\multicolumn{1}{c} {$ \varphi_{4}$} &
\multicolumn{1}{c} {$ \varphi_{5}$} &
\multicolumn{1}{c} {$ \varphi_{6}$} &
\multicolumn{1}{c} {$ \varphi_{7}$} &
\multicolumn{1}{c} {$ \varphi_{8}$} & 
\multicolumn{1}{c} {$ \varphi_{9}$} & 
\multicolumn{1}{c} {$ \varphi_{10}$} \\
\cline{3-12}
& previous & $\top$ & $\top$ & $\bot$ & $\top$ & $\bot$ & $\bot$ & $\top$ & $\bot$ & $\bot$ & $\top$ \\
\cline{3-12}
\\
\multicolumn{1}{c}{} &
\multicolumn{1}{c}{} &
\multicolumn{1}{c} {$ \varphi_{1}$} &
\multicolumn{1}{c} {$ \varphi_{2}$} &
\multicolumn{1}{c} {$ \varphi_{3}$} &
\multicolumn{1}{c} {$ \varphi_{4}$} &
\multicolumn{1}{c} {$ \varphi_{5}$} &
\multicolumn{1}{c} {$ \varphi_{6}$} &
\multicolumn{1}{c} {$ \varphi_{7}$} &
\multicolumn{1}{c} {$ \varphi_{8}$} & 
\multicolumn{1}{c} {$ \varphi_{9}$} & 
\multicolumn{1}{c} {$ \varphi_{10}$} \\
\cline{3-12}
& now & $\LTLalwaysbeen$ & $\rightarrow$ & $S$ & $\LTLonce$ & $r$ & $q$ & $\top$ & $\top$ & $\top$ & $\bot$ \\
\cline{3-12}
\end{tabular}\\
\\
\\
\subitem \underline{Event 2, Iteration 5, $t = \langle p, q, r \rangle, e = q, i = 6$}\\
\\
Registers before:

\begin{tabular}{rr|c|c|c|c|c|c|c|c|c|c|} &
\multicolumn{1}{c}{} &
\multicolumn{1}{c} {$ \varphi_{1}$} &
\multicolumn{1}{c} {$ \varphi_{2}$} &
\multicolumn{1}{c} {$ \varphi_{3}$} &
\multicolumn{1}{c} {$ \varphi_{4}$} &
\multicolumn{1}{c} {$ \varphi_{5}$} &
\multicolumn{1}{c} {$ \varphi_{6}$} &
\multicolumn{1}{c} {$ \varphi_{7}$} &
\multicolumn{1}{c} {$ \varphi_{8}$} & 
\multicolumn{1}{c} {$ \varphi_{9}$} & 
\multicolumn{1}{c} {$ \varphi_{10}$} \\
\cline{3-12}
& previous & $\top$ & $\top$ & $\bot$ & $\top$ & $\bot$ & $\bot$ & $\top$ & $\bot$ & $\bot$ & $\top$ \\
\cline{3-12}
\\
\multicolumn{1}{c}{} &
\multicolumn{1}{c}{} &
\multicolumn{1}{c} {$ \varphi_{1}$} &
\multicolumn{1}{c} {$ \varphi_{2}$} &
\multicolumn{1}{c} {$ \varphi_{3}$} &
\multicolumn{1}{c} {$ \varphi_{4}$} &
\multicolumn{1}{c} {$ \varphi_{5}$} &
\multicolumn{1}{c} {$ \varphi_{6}$} &
\multicolumn{1}{c} {$ \varphi_{7}$} &
\multicolumn{1}{c} {$ \varphi_{8}$} & 
\multicolumn{1}{c} {$ \varphi_{9}$} & 
\multicolumn{1}{c} {$ \varphi_{10}$} \\
\cline{3-12}
& now & $\LTLalwaysbeen$ & $\rightarrow$ & $S$ & $\LTLonce$ & $r$ & $q$ & $\top$ & $\top$ & $\top$ & $\bot$ \\
\cline{3-12}
\end{tabular}\\
\\
\\
Element $now[6]$ is evaluated next, it corresponds to the literal formula $q$.  The literal matches the event being evaluated so the element is assigned $\top$ by rule 10.\\
\\
\newpage
Registers after:

\begin{tabular}{rr|c|c|c|c|c|c|c|c|c|c|} &
\multicolumn{1}{c}{} &
\multicolumn{1}{c} {$ \varphi_{1}$} &
\multicolumn{1}{c} {$ \varphi_{2}$} &
\multicolumn{1}{c} {$ \varphi_{3}$} &
\multicolumn{1}{c} {$ \varphi_{4}$} &
\multicolumn{1}{c} {$ \varphi_{5}$} &
\multicolumn{1}{c} {$ \varphi_{6}$} &
\multicolumn{1}{c} {$ \varphi_{7}$} &
\multicolumn{1}{c} {$ \varphi_{8}$} & 
\multicolumn{1}{c} {$ \varphi_{9}$} & 
\multicolumn{1}{c} {$ \varphi_{10}$} \\
\cline{3-12}
& previous & $\top$ & $\top$ & $\bot$ & $\top$ & $\bot$ & $\bot$ & $\top$ & $\bot$ & $\bot$ & $\top$ \\
\cline{3-12}
\\
\multicolumn{1}{c}{} &
\multicolumn{1}{c}{} &
\multicolumn{1}{c} {$ \varphi_{1}$} &
\multicolumn{1}{c} {$ \varphi_{2}$} &
\multicolumn{1}{c} {$ \varphi_{3}$} &
\multicolumn{1}{c} {$ \varphi_{4}$} &
\multicolumn{1}{c} {$ \varphi_{5}$} &
\multicolumn{1}{c} {$ \varphi_{6}$} &
\multicolumn{1}{c} {$ \varphi_{7}$} &
\multicolumn{1}{c} {$ \varphi_{8}$} & 
\multicolumn{1}{c} {$ \varphi_{9}$} & 
\multicolumn{1}{c} {$ \varphi_{10}$} \\
\cline{3-12}
& now & $\LTLalwaysbeen$ & $\rightarrow$ & $S$ & $\LTLonce$ & $r$ & $\top$ & $\top$ & $\top$ & $\top$ & $\bot$ \\
\cline{3-12}
\end{tabular}\\
\\
\\
\subitem \underline{Event 2, Iteration 6, $t = \langle p, q, r \rangle, e = q, i = 5$}\\
\\
Registers before:

\begin{tabular}{rr|c|c|c|c|c|c|c|c|c|c|} &
\multicolumn{1}{c}{} &
\multicolumn{1}{c} {$ \varphi_{1}$} &
\multicolumn{1}{c} {$ \varphi_{2}$} &
\multicolumn{1}{c} {$ \varphi_{3}$} &
\multicolumn{1}{c} {$ \varphi_{4}$} &
\multicolumn{1}{c} {$ \varphi_{5}$} &
\multicolumn{1}{c} {$ \varphi_{6}$} &
\multicolumn{1}{c} {$ \varphi_{7}$} &
\multicolumn{1}{c} {$ \varphi_{8}$} & 
\multicolumn{1}{c} {$ \varphi_{9}$} & 
\multicolumn{1}{c} {$ \varphi_{10}$} \\
\cline{3-12}
& previous & $\top$ & $\top$ & $\bot$ & $\top$ & $\bot$ & $\bot$ & $\top$ & $\bot$ & $\bot$ & $\top$ \\
\cline{3-12}
\\
\multicolumn{1}{c}{} &
\multicolumn{1}{c}{} &
\multicolumn{1}{c} {$ \varphi_{1}$} &
\multicolumn{1}{c} {$ \varphi_{2}$} &
\multicolumn{1}{c} {$ \varphi_{3}$} &
\multicolumn{1}{c} {$ \varphi_{4}$} &
\multicolumn{1}{c} {$ \varphi_{5}$} &
\multicolumn{1}{c} {$ \varphi_{6}$} &
\multicolumn{1}{c} {$ \varphi_{7}$} &
\multicolumn{1}{c} {$ \varphi_{8}$} & 
\multicolumn{1}{c} {$ \varphi_{9}$} & 
\multicolumn{1}{c} {$ \varphi_{10}$} \\
\cline{3-12}
& now & $\LTLalwaysbeen$ & $\rightarrow$ & $S$ & $\LTLonce$ & $r$ & $\top$ & $\top$ & $\top$ & $\top$ & $\bot$ \\
\cline{3-12}
\end{tabular}\\
\\
\\
Element $now[5]$ is another literal formula, it is evaluated by applying rule 10.  The element is assigned $\bot$ because the literal does not match the event.\\
\\
Registers after:

\begin{tabular}{rr|c|c|c|c|c|c|c|c|c|c|} &
\multicolumn{1}{c}{} &
\multicolumn{1}{c} {$ \varphi_{1}$} &
\multicolumn{1}{c} {$ \varphi_{2}$} &
\multicolumn{1}{c} {$ \varphi_{3}$} &
\multicolumn{1}{c} {$ \varphi_{4}$} &
\multicolumn{1}{c} {$ \varphi_{5}$} &
\multicolumn{1}{c} {$ \varphi_{6}$} &
\multicolumn{1}{c} {$ \varphi_{7}$} &
\multicolumn{1}{c} {$ \varphi_{8}$} & 
\multicolumn{1}{c} {$ \varphi_{9}$} & 
\multicolumn{1}{c} {$ \varphi_{10}$} \\
\cline{3-12}
& previous & $\top$ & $\top$ & $\bot$ & $\top$ & $\bot$ & $\bot$ & $\top$ & $\bot$ & $\bot$ & $\top$ \\
\cline{3-12}
\\
\multicolumn{1}{c}{} &
\multicolumn{1}{c}{} &
\multicolumn{1}{c} {$ \varphi_{1}$} &
\multicolumn{1}{c} {$ \varphi_{2}$} &
\multicolumn{1}{c} {$ \varphi_{3}$} &
\multicolumn{1}{c} {$ \varphi_{4}$} &
\multicolumn{1}{c} {$ \varphi_{5}$} &
\multicolumn{1}{c} {$ \varphi_{6}$} &
\multicolumn{1}{c} {$ \varphi_{7}$} &
\multicolumn{1}{c} {$ \varphi_{8}$} & 
\multicolumn{1}{c} {$ \varphi_{9}$} & 
\multicolumn{1}{c} {$ \varphi_{10}$} \\
\cline{3-12}
& now & $\LTLalwaysbeen$ & $\rightarrow$ & $S$ & $\LTLonce$ & $\bot$ & $\top$ & $\top$ & $\top$ & $\top$ & $\bot$ \\
\cline{3-12}
\end{tabular}\\
\\
\\
\subitem \underline{Event 2, Iteration 7, $t = \langle p, q, r \rangle, e = q, i = 4, j = 7$}\\
\\
Registers before:

\begin{tabular}{rr|c|c|c|c|c|c|c|c|c|c|} &
\multicolumn{1}{c}{} &
\multicolumn{1}{c} {$ \varphi_{1}$} &
\multicolumn{1}{c} {$ \varphi_{2}$} &
\multicolumn{1}{c} {$ \varphi_{3}$} &
\multicolumn{1}{c} {$ \varphi_{4}$} &
\multicolumn{1}{c} {$ \varphi_{5}$} &
\multicolumn{1}{c} {$ \varphi_{6}$} &
\multicolumn{1}{c} {$ \varphi_{7}$} &
\multicolumn{1}{c} {$ \varphi_{8}$} & 
\multicolumn{1}{c} {$ \varphi_{9}$} & 
\multicolumn{1}{c} {$ \varphi_{10}$} \\
\cline{3-12}
& previous & $\top$ & $\top$ & $\bot$ & $\top$ & $\bot$ & $\bot$ & $\top$ & $\bot$ & $\bot$ & $\top$ \\
\cline{3-12}
\\
\multicolumn{1}{c}{} &
\multicolumn{1}{c}{} &
\multicolumn{1}{c} {$ \varphi_{1}$} &
\multicolumn{1}{c} {$ \varphi_{2}$} &
\multicolumn{1}{c} {$ \varphi_{3}$} &
\multicolumn{1}{c} {$ \varphi_{4}$} &
\multicolumn{1}{c} {$ \varphi_{5}$} &
\multicolumn{1}{c} {$ \varphi_{6}$} &
\multicolumn{1}{c} {$ \varphi_{7}$} &
\multicolumn{1}{c} {$ \varphi_{8}$} & 
\multicolumn{1}{c} {$ \varphi_{9}$} & 
\multicolumn{1}{c} {$ \varphi_{10}$} \\
\cline{3-12}
& now & $\LTLalwaysbeen$ & $\rightarrow$ & $S$ & $\LTLonce$ & $\bot$ & $\top$ & $\top$ & $\top$ & $\top$ & $\bot$ \\
\cline{3-12}
\end{tabular}\\
\\
\\
Element $now[4]$ corresponds to a formula with once as the operator.  It is assigned $\top$ by rule 17 when the left operand, indexed by $j$, is $\top$ for the present event or the formula evaluated $\top$ for a previous event.  In this case it is assigned $\top$ because the left operand, $now[7]$, is $\top$.\\
\\
Registers after:

\begin{tabular}{rr|c|c|c|c|c|c|c|c|c|c|} &
\multicolumn{1}{c}{} &
\multicolumn{1}{c} {$ \varphi_{1}$} &
\multicolumn{1}{c} {$ \varphi_{2}$} &
\multicolumn{1}{c} {$ \varphi_{3}$} &
\multicolumn{1}{c} {$ \varphi_{4}$} &
\multicolumn{1}{c} {$ \varphi_{5}$} &
\multicolumn{1}{c} {$ \varphi_{6}$} &
\multicolumn{1}{c} {$ \varphi_{7}$} &
\multicolumn{1}{c} {$ \varphi_{8}$} & 
\multicolumn{1}{c} {$ \varphi_{9}$} & 
\multicolumn{1}{c} {$ \varphi_{10}$} \\
\cline{3-12}
& previous & $\top$ & $\top$ & $\bot$ & $\top$ & $\bot$ & $\bot$ & $\top$ & $\bot$ & $\bot$ & $\top$ \\
\cline{3-12}
\\
\multicolumn{1}{c}{} &
\multicolumn{1}{c}{} &
\multicolumn{1}{c} {$ \varphi_{1}$} &
\multicolumn{1}{c} {$ \varphi_{2}$} &
\multicolumn{1}{c} {$ \varphi_{3}$} &
\multicolumn{1}{c} {$ \varphi_{4}$} &
\multicolumn{1}{c} {$ \varphi_{5}$} &
\multicolumn{1}{c} {$ \varphi_{6}$} &
\multicolumn{1}{c} {$ \varphi_{7}$} &
\multicolumn{1}{c} {$ \varphi_{8}$} & 
\multicolumn{1}{c} {$ \varphi_{9}$} & 
\multicolumn{1}{c} {$ \varphi_{10}$} \\
\cline{3-12}
& now & $\LTLalwaysbeen$ & $\rightarrow$ & $S$ & $\top$ & $\bot$ & $\top$ & $\top$ & $\top$ & $\top$ & $\bot$ \\
\cline{3-12}
\end{tabular}\\
\\
\\
\subitem \underline{Event 2, Iteration 8, $t = \langle p, q, r \rangle, e = q, i = 3, j = 5, k = 6$}\\
\\
Registers before:

\begin{tabular}{rr|c|c|c|c|c|c|c|c|c|c|} &
\multicolumn{1}{c}{} &
\multicolumn{1}{c} {$ \varphi_{1}$} &
\multicolumn{1}{c} {$ \varphi_{2}$} &
\multicolumn{1}{c} {$ \varphi_{3}$} &
\multicolumn{1}{c} {$ \varphi_{4}$} &
\multicolumn{1}{c} {$ \varphi_{5}$} &
\multicolumn{1}{c} {$ \varphi_{6}$} &
\multicolumn{1}{c} {$ \varphi_{7}$} &
\multicolumn{1}{c} {$ \varphi_{8}$} & 
\multicolumn{1}{c} {$ \varphi_{9}$} & 
\multicolumn{1}{c} {$ \varphi_{10}$} \\
\cline{3-12}
& previous & $\top$ & $\top$ & $\bot$ & $\top$ & $\bot$ & $\bot$ & $\top$ & $\bot$ & $\bot$ & $\top$ \\
\cline{3-12}
\\
\multicolumn{1}{c}{} &
\multicolumn{1}{c}{} &
\multicolumn{1}{c} {$ \varphi_{1}$} &
\multicolumn{1}{c} {$ \varphi_{2}$} &
\multicolumn{1}{c} {$ \varphi_{3}$} &
\multicolumn{1}{c} {$ \varphi_{4}$} &
\multicolumn{1}{c} {$ \varphi_{5}$} &
\multicolumn{1}{c} {$ \varphi_{6}$} &
\multicolumn{1}{c} {$ \varphi_{7}$} &
\multicolumn{1}{c} {$ \varphi_{8}$} & 
\multicolumn{1}{c} {$ \varphi_{9}$} & 
\multicolumn{1}{c} {$ \varphi_{10}$} \\
\cline{3-12}
& now & $\LTLalwaysbeen$ & $\rightarrow$ & $S$ & $\top$ & $\bot$ & $\top$ & $\top$ & $\top$ & $\top$ & $\bot$ \\
\cline{3-12}
\end{tabular}\\
\\
\\
The formula that element $now[3]$ corresponds to has a since operator at its root.  Rule 18 defines the since operator semantics.  In this case $now[3]$ is assigned $\top$ because the right operand, $now[7]$ is $\top$.\\
\\
Registers after:

\begin{tabular}{rr|c|c|c|c|c|c|c|c|c|c|} &
\multicolumn{1}{c}{} &
\multicolumn{1}{c} {$ \varphi_{1}$} &
\multicolumn{1}{c} {$ \varphi_{2}$} &
\multicolumn{1}{c} {$ \varphi_{3}$} &
\multicolumn{1}{c} {$ \varphi_{4}$} &
\multicolumn{1}{c} {$ \varphi_{5}$} &
\multicolumn{1}{c} {$ \varphi_{6}$} &
\multicolumn{1}{c} {$ \varphi_{7}$} &
\multicolumn{1}{c} {$ \varphi_{8}$} & 
\multicolumn{1}{c} {$ \varphi_{9}$} & 
\multicolumn{1}{c} {$ \varphi_{10}$} \\
\cline{3-12}
& previous & $\top$ & $\top$ & $\bot$ & $\top$ & $\bot$ & $\bot$ & $\top$ & $\bot$ & $\bot$ & $\top$ \\
\cline{3-12}
\\
\multicolumn{1}{c}{} &
\multicolumn{1}{c}{} &
\multicolumn{1}{c} {$ \varphi_{1}$} &
\multicolumn{1}{c} {$ \varphi_{2}$} &
\multicolumn{1}{c} {$ \varphi_{3}$} &
\multicolumn{1}{c} {$ \varphi_{4}$} &
\multicolumn{1}{c} {$ \varphi_{5}$} &
\multicolumn{1}{c} {$ \varphi_{6}$} &
\multicolumn{1}{c} {$ \varphi_{7}$} &
\multicolumn{1}{c} {$ \varphi_{8}$} & 
\multicolumn{1}{c} {$ \varphi_{9}$} & 
\multicolumn{1}{c} {$ \varphi_{10}$} \\
\cline{3-12}
& now & $\LTLalwaysbeen$ & $\rightarrow$ & $\top$ & $\top$ & $\bot$ & $\top$ & $\top$ & $\top$ & $\top$ & $\bot$ \\
\cline{3-12}
\end{tabular}\\
\\
\\
\newpage
\subitem \underline{Event 2, Iteration 9, $t = \langle p, q, r \rangle, e = q, i = 2, j = 3, k = 4$}\\
\\
Registers before:

\begin{tabular}{rr|c|c|c|c|c|c|c|c|c|c|} &
\multicolumn{1}{c}{} &
\multicolumn{1}{c} {$ \varphi_{1}$} &
\multicolumn{1}{c} {$ \varphi_{2}$} &
\multicolumn{1}{c} {$ \varphi_{3}$} &
\multicolumn{1}{c} {$ \varphi_{4}$} &
\multicolumn{1}{c} {$ \varphi_{5}$} &
\multicolumn{1}{c} {$ \varphi_{6}$} &
\multicolumn{1}{c} {$ \varphi_{7}$} &
\multicolumn{1}{c} {$ \varphi_{8}$} & 
\multicolumn{1}{c} {$ \varphi_{9}$} & 
\multicolumn{1}{c} {$ \varphi_{10}$} \\
\cline{3-12}
& previous & $\top$ & $\top$ & $\bot$ & $\top$ & $\bot$ & $\bot$ & $\top$ & $\bot$ & $\bot$ & $\top$ \\
\cline{3-12}\\
\multicolumn{1}{c}{} &
\multicolumn{1}{c}{} &
\multicolumn{1}{c} {$ \varphi_{1}$} &
\multicolumn{1}{c} {$ \varphi_{2}$} &
\multicolumn{1}{c} {$ \varphi_{3}$} &
\multicolumn{1}{c} {$ \varphi_{4}$} &
\multicolumn{1}{c} {$ \varphi_{5}$} &
\multicolumn{1}{c} {$ \varphi_{6}$} &
\multicolumn{1}{c} {$ \varphi_{7}$} &
\multicolumn{1}{c} {$ \varphi_{8}$} & 
\multicolumn{1}{c} {$ \varphi_{9}$} & 
\multicolumn{1}{c} {$ \varphi_{10}$} \\
\cline{3-12}
& now & $\LTLalwaysbeen$ & $\rightarrow$ & $\top$ & $\top$ & $\bot$ & $\top$ & $\top$ & $\top$ & $\top$ & $\bot$ \\
\cline{3-12}
\end{tabular}\\
\\
\\
The next element to evaluate is $now[2]$, according to the implies operator semantics of rule 5, the element is assigned $\top$ because the left and right operators, elements $now[3]$ and $now[4]$, both evaluated to $\top$.\\
\\
Registers after:

\begin{tabular}{rr|c|c|c|c|c|c|c|c|c|c|} &
\multicolumn{1}{c}{} &
\multicolumn{1}{c} {$ \varphi_{1}$} &
\multicolumn{1}{c} {$ \varphi_{2}$} &
\multicolumn{1}{c} {$ \varphi_{3}$} &
\multicolumn{1}{c} {$ \varphi_{4}$} &
\multicolumn{1}{c} {$ \varphi_{5}$} &
\multicolumn{1}{c} {$ \varphi_{6}$} &
\multicolumn{1}{c} {$ \varphi_{7}$} &
\multicolumn{1}{c} {$ \varphi_{8}$} & 
\multicolumn{1}{c} {$ \varphi_{9}$} & 
\multicolumn{1}{c} {$ \varphi_{10}$} \\
\cline{3-12}
& previous & $\top$ & $\top$ & $\bot$ & $\top$ & $\bot$ & $\bot$ & $\top$ & $\bot$ & $\bot$ & $\top$ \\
\cline{3-12}
\\
\multicolumn{1}{c}{} &
\multicolumn{1}{c}{} &
\multicolumn{1}{c} {$ \varphi_{1}$} &
\multicolumn{1}{c} {$ \varphi_{2}$} &
\multicolumn{1}{c} {$ \varphi_{3}$} &
\multicolumn{1}{c} {$ \varphi_{4}$} &
\multicolumn{1}{c} {$ \varphi_{5}$} &
\multicolumn{1}{c} {$ \varphi_{6}$} &
\multicolumn{1}{c} {$ \varphi_{7}$} &
\multicolumn{1}{c} {$ \varphi_{8}$} & 
\multicolumn{1}{c} {$ \varphi_{9}$} & 
\multicolumn{1}{c} {$ \varphi_{10}$} \\
\cline{3-12}
& now & $\LTLalwaysbeen$ & $\top$ & $\top$ & $\top$ & $\bot$ & $\top$ & $\top$ & $\top$ & $\top$ & $\bot$ \\
\cline{3-12}
\end{tabular}\\
\\
\\
\subitem \underline{Event 2, Iteration 10, $t = \langle p, q, r \rangle, e = q, i = 1, j = 2$}\\
\\
Registers before:

\begin{tabular}{rr|c|c|c|c|c|c|c|c|c|c|} &
\multicolumn{1}{c}{} &
\multicolumn{1}{c} {$ \varphi_{1}$} &
\multicolumn{1}{c} {$ \varphi_{2}$} &
\multicolumn{1}{c} {$ \varphi_{3}$} &
\multicolumn{1}{c} {$ \varphi_{4}$} &
\multicolumn{1}{c} {$ \varphi_{5}$} &
\multicolumn{1}{c} {$ \varphi_{6}$} &
\multicolumn{1}{c} {$ \varphi_{7}$} &
\multicolumn{1}{c} {$ \varphi_{8}$} & 
\multicolumn{1}{c} {$ \varphi_{9}$} & 
\multicolumn{1}{c} {$ \varphi_{10}$} \\
\cline{3-12}
& previous & $\top$ & $\top$ & $\bot$ & $\top$ & $\bot$ & $\bot$ & $\top$ & $\bot$ & $\bot$ & $\top$ \\
\cline{3-12}
\\
\multicolumn{1}{c}{} &
\multicolumn{1}{c}{} &
\multicolumn{1}{c} {$ \varphi_{1}$} &
\multicolumn{1}{c} {$ \varphi_{2}$} &
\multicolumn{1}{c} {$ \varphi_{3}$} &
\multicolumn{1}{c} {$ \varphi_{4}$} &
\multicolumn{1}{c} {$ \varphi_{5}$} &
\multicolumn{1}{c} {$ \varphi_{6}$} &
\multicolumn{1}{c} {$ \varphi_{7}$} &
\multicolumn{1}{c} {$ \varphi_{8}$} & 
\multicolumn{1}{c} {$ \varphi_{9}$} & 
\multicolumn{1}{c} {$ \varphi_{10}$} \\
\cline{3-12}
& now & $\LTLalwaysbeen$ & $\top$ & $\top$ & $\top$ & $\bot$ & $\top$ & $\top$ & $\top$ & $\top$ & $\bot$ \\
\cline{3-12}
\end{tabular}\\
\\
\\
The event $q$ is evaluated over the last element, $now[1]$.  The corresponding formula has the always been operator at its root.  The semantics are defined by rule 16 and state the element is assigned $\top$ when the left operand, $now[2]$ is $\top$ and $previous[1]$ is $\top$.  Both elements are $\top$, therefore $now[1]$ is assigned $\top$.\\
\\
\newpage
Registers after:

\begin{tabular}{rr|c|c|c|c|c|c|c|c|c|c|} &
\multicolumn{1}{c}{} &
\multicolumn{1}{c} {$ \varphi_{1}$} &
\multicolumn{1}{c} {$ \varphi_{2}$} &
\multicolumn{1}{c} {$ \varphi_{3}$} &
\multicolumn{1}{c} {$ \varphi_{4}$} &
\multicolumn{1}{c} {$ \varphi_{5}$} &
\multicolumn{1}{c} {$ \varphi_{6}$} &
\multicolumn{1}{c} {$ \varphi_{7}$} &
\multicolumn{1}{c} {$ \varphi_{8}$} & 
\multicolumn{1}{c} {$ \varphi_{9}$} & 
\multicolumn{1}{c} {$ \varphi_{10}$} \\
\cline{3-12}
& previous & $\top$ & $\top$ & $\bot$ & $\top$ & $\bot$ & $\bot$ & $\top$ & $\bot$ & $\bot$ & $\top$ \\
\cline{3-12}
\\
\multicolumn{1}{c}{} &
\multicolumn{1}{c}{} &
\multicolumn{1}{c} {$ \varphi_{1}$} &
\multicolumn{1}{c} {$ \varphi_{2}$} &
\multicolumn{1}{c} {$ \varphi_{3}$} &
\multicolumn{1}{c} {$ \varphi_{4}$} &
\multicolumn{1}{c} {$ \varphi_{5}$} &
\multicolumn{1}{c} {$ \varphi_{6}$} &
\multicolumn{1}{c} {$ \varphi_{7}$} &
\multicolumn{1}{c} {$ \varphi_{8}$} & 
\multicolumn{1}{c} {$ \varphi_{9}$} & 
\multicolumn{1}{c} {$ \varphi_{10}$} \\
\cline{3-12}
& now & $\top$ & $\top$ & $\top$ & $\top$ & $\bot$ & $\top$ & $\top$ & $\top$ & $\top$ & $\bot$ \\
\cline{3-12}
\end{tabular}\\
\\
\\
\textbf{\item Evaluation Phase - Step 2 repeated}\\
\\
Register before:\\

\begin{tabular}{rr|c|c|c|c|c|c|c|c|c|c|} &
\multicolumn{1}{c}{} &
\multicolumn{1}{c} {$ \varphi_{1}$} &
\multicolumn{1}{c} {$ \varphi_{2}$} &
\multicolumn{1}{c} {$ \varphi_{3}$} &
\multicolumn{1}{c} {$ \varphi_{4}$} &
\multicolumn{1}{c} {$ \varphi_{5}$} &
\multicolumn{1}{c} {$ \varphi_{6}$} &
\multicolumn{1}{c} {$ \varphi_{7}$} &
\multicolumn{1}{c} {$ \varphi_{8}$} & 
\multicolumn{1}{c} {$ \varphi_{9}$} & 
\multicolumn{1}{c} {$ \varphi_{10}$} \\
\cline{3-12}
& previous & $\top$ & $\top$ & $\bot$ & $\top$ & $\bot$ & $\bot$ & $\top$ & $\bot$ & $\bot$ & $\top$ \\
\cline{3-12}
\\
\multicolumn{1}{c}{} &
\multicolumn{1}{c}{} &
\multicolumn{1}{c} {$ \varphi_{1}$} &
\multicolumn{1}{c} {$ \varphi_{2}$} &
\multicolumn{1}{c} {$ \varphi_{3}$} &
\multicolumn{1}{c} {$ \varphi_{4}$} &
\multicolumn{1}{c} {$ \varphi_{5}$} &
\multicolumn{1}{c} {$ \varphi_{6}$} &
\multicolumn{1}{c} {$ \varphi_{7}$} &
\multicolumn{1}{c} {$ \varphi_{8}$} & 
\multicolumn{1}{c} {$ \varphi_{9}$} & 
\multicolumn{1}{c} {$ \varphi_{10}$} \\
\cline{3-12}
& now & $\top$ & $\top$ & $\top$ & $\top$ & $\bot$ & $\top$ & $\top$ & $\top$ & $\top$ & $\bot$ \\
\cline{3-12}
\end{tabular}\\
\\
\\
The $q$ event has been completely evaluated so step 1 of evaluation is finished.  Step 2 is to overwrite all the values in the \textit{previous} register with those from the \textit{now} register.\\
\\
Registers after:

\begin{tabular}{rr|c|c|c|c|c|c|c|c|c|c|} &
\multicolumn{1}{c}{} &
\multicolumn{1}{c} {$ \varphi_{1}$} &
\multicolumn{1}{c} {$ \varphi_{2}$} &
\multicolumn{1}{c} {$ \varphi_{3}$} &
\multicolumn{1}{c} {$ \varphi_{4}$} &
\multicolumn{1}{c} {$ \varphi_{5}$} &
\multicolumn{1}{c} {$ \varphi_{6}$} &
\multicolumn{1}{c} {$ \varphi_{7}$} &
\multicolumn{1}{c} {$ \varphi_{8}$} & 
\multicolumn{1}{c} {$ \varphi_{9}$} & 
\multicolumn{1}{c} {$ \varphi_{10}$} \\
\cline{3-12}
& previous & $\top$ & $\top$ & $\top$ & $\top$ & $\bot$ & $\top$ & $\top$ & $\top$ & $\top$ & $\bot$ \\
\cline{3-12}
\\
\multicolumn{1}{c}{} &
\multicolumn{1}{c}{} &
\multicolumn{1}{c} {$ \varphi_{1}$} &
\multicolumn{1}{c} {$ \varphi_{2}$} &
\multicolumn{1}{c} {$ \varphi_{3}$} &
\multicolumn{1}{c} {$ \varphi_{4}$} &
\multicolumn{1}{c} {$ \varphi_{5}$} &
\multicolumn{1}{c} {$ \varphi_{6}$} &
\multicolumn{1}{c} {$ \varphi_{7}$} &
\multicolumn{1}{c} {$ \varphi_{8}$} & 
\multicolumn{1}{c} {$ \varphi_{9}$} & 
\multicolumn{1}{c} {$ \varphi_{10}$} \\
\cline{3-12}
& now & $\top$ & $\top$ & $\top$ & $\top$ & $\bot$ & $\top$ & $\top$ & $\top$ & $\top$ & $\bot$ \\
\cline{3-12}
\end{tabular}\\

\newpage
\textbf{\item Evaluation Phase - Step 3 repeated}\\\\Evaluation step 3 is to repeat steps 1 and 2 for another event in the trace.  The algorithm traverses the trace from the earliest to the latest event, so the $r$ event will be evaluated next.\\

% =====
% Event 3
% =====
\subitem \underline{Event 3, Iteration 1: $t = \langle p, q, r \rangle, e = r, i = 10$}\\
\\
Registers before:

\begin{tabular}{rr|c|c|c|c|c|c|c|c|c|c|} &
\multicolumn{1}{c}{} &
\multicolumn{1}{c} {$ \varphi_{1}$} &
\multicolumn{1}{c} {$ \varphi_{2}$} &
\multicolumn{1}{c} {$ \varphi_{3}$} &
\multicolumn{1}{c} {$ \varphi_{4}$} &
\multicolumn{1}{c} {$ \varphi_{5}$} &
\multicolumn{1}{c} {$ \varphi_{6}$} &
\multicolumn{1}{c} {$ \varphi_{7}$} &
\multicolumn{1}{c} {$ \varphi_{8}$} & 
\multicolumn{1}{c} {$ \varphi_{9}$} & 
\multicolumn{1}{c} {$ \varphi_{10}$} \\
\cline{3-12}
& previous & $\top$ & $\top$ & $\top$ & $\top$ & $\bot$ & $\top$ & $\top$ & $\top$ & $\top$ & $\bot$ \\
\cline{3-12}
\\
\multicolumn{1}{c}{} &
\multicolumn{1}{c}{} &
\multicolumn{1}{c} {$ \varphi_{1}$} &
\multicolumn{1}{c} {$ \varphi_{2}$} &
\multicolumn{1}{c} {$ \varphi_{3}$} &
\multicolumn{1}{c} {$ \varphi_{4}$} &
\multicolumn{1}{c} {$ \varphi_{5}$} &
\multicolumn{1}{c} {$ \varphi_{6}$} &
\multicolumn{1}{c} {$ \varphi_{7}$} &
\multicolumn{1}{c} {$ \varphi_{8}$} & 
\multicolumn{1}{c} {$ \varphi_{9}$} & 
\multicolumn{1}{c} {$ \varphi_{10}$} \\
\cline{3-12}
& now & $\LTLalwaysbeen$ & $\rightarrow$ & $S$ & $\LTLonce$ & $r$ & $q$ & $\rightarrow$ & $q$ & $\LTLprevious$ & $p$ \\
\cline{3-12}
\end{tabular}\\
\\
\\
The $now[10]$ element is assigned $\bot$ by rule 10 because the event $r$ does not match the literal formula $p$.\\
\\
Registers after:

\begin{tabular}{rr|c|c|c|c|c|c|c|c|c|c|} &
\multicolumn{1}{c}{} &
\multicolumn{1}{c} {$ \varphi_{1}$} &
\multicolumn{1}{c} {$ \varphi_{2}$} &
\multicolumn{1}{c} {$ \varphi_{3}$} &
\multicolumn{1}{c} {$ \varphi_{4}$} &
\multicolumn{1}{c} {$ \varphi_{5}$} &
\multicolumn{1}{c} {$ \varphi_{6}$} &
\multicolumn{1}{c} {$ \varphi_{7}$} &
\multicolumn{1}{c} {$ \varphi_{8}$} & 
\multicolumn{1}{c} {$ \varphi_{9}$} & 
\multicolumn{1}{c} {$ \varphi_{10}$} \\
\cline{3-12}
& previous & $\top$ & $\top$ & $\top$ & $\top$ & $\bot$ & $\top$ & $\top$ & $\top$ & $\top$ & $\bot$ \\
\cline{3-12}
\\
\multicolumn{1}{c}{} &
\multicolumn{1}{c}{} &
\multicolumn{1}{c} {$ \varphi_{1}$} &
\multicolumn{1}{c} {$ \varphi_{2}$} &
\multicolumn{1}{c} {$ \varphi_{3}$} &
\multicolumn{1}{c} {$ \varphi_{4}$} &
\multicolumn{1}{c} {$ \varphi_{5}$} &
\multicolumn{1}{c} {$ \varphi_{6}$} &
\multicolumn{1}{c} {$ \varphi_{7}$} &
\multicolumn{1}{c} {$ \varphi_{8}$} & 
\multicolumn{1}{c} {$ \varphi_{9}$} & 
\multicolumn{1}{c} {$ \varphi_{10}$} \\
\cline{3-12}
& now & $\LTLalwaysbeen$ & $\rightarrow$ & $S$ & $\LTLonce$ & $r$ & $q$ & $\rightarrow$ & $q$ & $\LTLprevious$ & $\bot$ \\
\cline{3-12}
\end{tabular}\\
\\
\\
\subitem \underline{Event 3, Iteration 2: $t = \langle p, q, r \rangle, e = r, i = 9$, j = 10}\\
\\
Registers before:

\begin{tabular}{rr|c|c|c|c|c|c|c|c|c|c|} &
\multicolumn{1}{c}{} &
\multicolumn{1}{c} {$ \varphi_{1}$} &
\multicolumn{1}{c} {$ \varphi_{2}$} &
\multicolumn{1}{c} {$ \varphi_{3}$} &
\multicolumn{1}{c} {$ \varphi_{4}$} &
\multicolumn{1}{c} {$ \varphi_{5}$} &
\multicolumn{1}{c} {$ \varphi_{6}$} &
\multicolumn{1}{c} {$ \varphi_{7}$} &
\multicolumn{1}{c} {$ \varphi_{8}$} & 
\multicolumn{1}{c} {$ \varphi_{9}$} & 
\multicolumn{1}{c} {$ \varphi_{10}$} \\
\cline{3-12}
& previous & $\top$ & $\top$ & $\top$ & $\top$ & $\bot$ & $\top$ & $\top$ & $\top$ & $\top$ & $\bot$ \\
\cline{3-12}
\\
\multicolumn{1}{c}{} &
\multicolumn{1}{c}{} &
\multicolumn{1}{c} {$ \varphi_{1}$} &
\multicolumn{1}{c} {$ \varphi_{2}$} &
\multicolumn{1}{c} {$ \varphi_{3}$} &
\multicolumn{1}{c} {$ \varphi_{4}$} &
\multicolumn{1}{c} {$ \varphi_{5}$} &
\multicolumn{1}{c} {$ \varphi_{6}$} &
\multicolumn{1}{c} {$ \varphi_{7}$} &
\multicolumn{1}{c} {$ \varphi_{8}$} & 
\multicolumn{1}{c} {$ \varphi_{9}$} & 
\multicolumn{1}{c} {$ \varphi_{10}$} \\
\cline{3-12}
& now & $\LTLalwaysbeen$ & $\rightarrow$ & $S$ & $\LTLonce$ & $r$ & $q$ & $\rightarrow$ & $q$ & $\LTLprevious$ & $\bot$ \\
\cline{3-12}
\end{tabular}\\
\\
\\
Element $now[9]$ is assigned $\bot$ by rule 15 because $previous[10]$ is $\bot$, indicating the left operand was $\bot$ for the previous event.\\
\\
Registers after:

\begin{tabular}{rr|c|c|c|c|c|c|c|c|c|c|} &
\multicolumn{1}{c}{} &
\multicolumn{1}{c} {$ \varphi_{1}$} &
\multicolumn{1}{c} {$ \varphi_{2}$} &
\multicolumn{1}{c} {$ \varphi_{3}$} &
\multicolumn{1}{c} {$ \varphi_{4}$} &
\multicolumn{1}{c} {$ \varphi_{5}$} &
\multicolumn{1}{c} {$ \varphi_{6}$} &
\multicolumn{1}{c} {$ \varphi_{7}$} &
\multicolumn{1}{c} {$ \varphi_{8}$} & 
\multicolumn{1}{c} {$ \varphi_{9}$} & 
\multicolumn{1}{c} {$ \varphi_{10}$} \\
\cline{3-12}
& previous & $\top$ & $\top$ & $\top$ & $\top$ & $\bot$ & $\top$ & $\top$ & $\top$ & $\top$ & $\bot$ \\
\cline{3-12}
\\
\multicolumn{1}{c}{} &
\multicolumn{1}{c}{} &
\multicolumn{1}{c} {$ \varphi_{1}$} &
\multicolumn{1}{c} {$ \varphi_{2}$} &
\multicolumn{1}{c} {$ \varphi_{3}$} &
\multicolumn{1}{c} {$ \varphi_{4}$} &
\multicolumn{1}{c} {$ \varphi_{5}$} &
\multicolumn{1}{c} {$ \varphi_{6}$} &
\multicolumn{1}{c} {$ \varphi_{7}$} &
\multicolumn{1}{c} {$ \varphi_{8}$} & 
\multicolumn{1}{c} {$ \varphi_{9}$} & 
\multicolumn{1}{c} {$ \varphi_{10}$} \\
\cline{3-12}
& now & $\LTLalwaysbeen$ & $\rightarrow$ & $S$ & $\LTLonce$ & $r$ & $q$ & $\rightarrow$ & $q$ & $\bot$ & $\bot$ \\
\cline{3-12}
\end{tabular}\\
\\
\\
\subitem \underline{Event 3, Iteration 3: $t = \langle p, q, r \rangle, e = r, i = 8$}\\
\\
Registers before:

\begin{tabular}{rr|c|c|c|c|c|c|c|c|c|c|} &
\multicolumn{1}{c}{} &
\multicolumn{1}{c} {$ \varphi_{1}$} &
\multicolumn{1}{c} {$ \varphi_{2}$} &
\multicolumn{1}{c} {$ \varphi_{3}$} &
\multicolumn{1}{c} {$ \varphi_{4}$} &
\multicolumn{1}{c} {$ \varphi_{5}$} &
\multicolumn{1}{c} {$ \varphi_{6}$} &
\multicolumn{1}{c} {$ \varphi_{7}$} &
\multicolumn{1}{c} {$ \varphi_{8}$} & 
\multicolumn{1}{c} {$ \varphi_{9}$} & 
\multicolumn{1}{c} {$ \varphi_{10}$} \\
\cline{3-12}
& previous & $\top$ & $\top$ & $\top$ & $\top$ & $\bot$ & $\top$ & $\top$ & $\top$ & $\top$ & $\bot$ \\
\cline{3-12}
\\
\multicolumn{1}{c}{} &
\multicolumn{1}{c}{} &
\multicolumn{1}{c} {$ \varphi_{1}$} &
\multicolumn{1}{c} {$ \varphi_{2}$} &
\multicolumn{1}{c} {$ \varphi_{3}$} &
\multicolumn{1}{c} {$ \varphi_{4}$} &
\multicolumn{1}{c} {$ \varphi_{5}$} &
\multicolumn{1}{c} {$ \varphi_{6}$} &
\multicolumn{1}{c} {$ \varphi_{7}$} &
\multicolumn{1}{c} {$ \varphi_{8}$} & 
\multicolumn{1}{c} {$ \varphi_{9}$} & 
\multicolumn{1}{c} {$ \varphi_{10}$} \\
\cline{3-12}
& now & $\LTLalwaysbeen$ & $\rightarrow$ & $S$ & $\LTLonce$ & $r$ & $q$ & $\rightarrow$ & $q$ & $\bot$ & $\bot$ \\
\cline{3-12}
\end{tabular}\\
\\
\\
Element $now[8]$ is assigned $\bot$ according to rule 10 of Definition \ref{def:AlgorithmicPastOperatorSemantics} because the literal formula $q$ does not match the event $r$ that is being evaluated.\\
\\
Registers after:

\begin{tabular}{rr|c|c|c|c|c|c|c|c|c|c|} &
\multicolumn{1}{c}{} &
\multicolumn{1}{c} {$ \varphi_{1}$} &
\multicolumn{1}{c} {$ \varphi_{2}$} &
\multicolumn{1}{c} {$ \varphi_{3}$} &
\multicolumn{1}{c} {$ \varphi_{4}$} &
\multicolumn{1}{c} {$ \varphi_{5}$} &
\multicolumn{1}{c} {$ \varphi_{6}$} &
\multicolumn{1}{c} {$ \varphi_{7}$} &
\multicolumn{1}{c} {$ \varphi_{8}$} & 
\multicolumn{1}{c} {$ \varphi_{9}$} & 
\multicolumn{1}{c} {$ \varphi_{10}$} \\
\cline{3-12}
& previous & $\top$ & $\top$ & $\top$ & $\top$ & $\bot$ & $\top$ & $\top$ & $\top$ & $\top$ & $\bot$ \\
\cline{3-12}
\\
\multicolumn{1}{c}{} &
\multicolumn{1}{c}{} &
\multicolumn{1}{c} {$ \varphi_{1}$} &
\multicolumn{1}{c} {$ \varphi_{2}$} &
\multicolumn{1}{c} {$ \varphi_{3}$} &
\multicolumn{1}{c} {$ \varphi_{4}$} &
\multicolumn{1}{c} {$ \varphi_{5}$} &
\multicolumn{1}{c} {$ \varphi_{6}$} &
\multicolumn{1}{c} {$ \varphi_{7}$} &
\multicolumn{1}{c} {$ \varphi_{8}$} & 
\multicolumn{1}{c} {$ \varphi_{9}$} & 
\multicolumn{1}{c} {$ \varphi_{10}$} \\
\cline{3-12}
& now & $\LTLalwaysbeen$ & $\rightarrow$ & $S$ & $\LTLonce$ & $r$ & $q$ & $\rightarrow$ & $\bot$ & $\bot$ & $\bot$ \\
\cline{3-12}
\end{tabular}\\
\\
\\
\newpage
\subitem \underline{Event 3, Iteration 4: $t = \langle p, q, r \rangle, e = r, i = 7, j = 8, k = 9$}\\
\\
Registers before:

\begin{tabular}{cc|c|c|c|c|c|c|c|c|c|c|} &
\multicolumn{1}{c}{} &
\multicolumn{1}{c} {$ \varphi_{1}$} &
\multicolumn{1}{c} {$ \varphi_{2}$} &
\multicolumn{1}{c} {$ \varphi_{3}$} &
\multicolumn{1}{c} {$ \varphi_{4}$} &
\multicolumn{1}{c} {$ \varphi_{5}$} &
\multicolumn{1}{c} {$ \varphi_{6}$} &
\multicolumn{1}{c} {$ \varphi_{7}$} &
\multicolumn{1}{c} {$ \varphi_{8}$} & 
\multicolumn{1}{c} {$ \varphi_{9}$} & 
\multicolumn{1}{c} {$ \varphi_{10}$} \\
\cline{3-12}
& previous & $\top$ & $\top$ & $\top$ & $\top$ & $\bot$ & $\top$ & $\top$ & $\top$ & $\top$ & $\bot$ \\
\cline{3-12}
\\
\multicolumn{1}{c}{} &
\multicolumn{1}{c}{} &
\multicolumn{1}{c} {$ \varphi_{1}$} &
\multicolumn{1}{c} {$ \varphi_{2}$} &
\multicolumn{1}{c} {$ \varphi_{3}$} &
\multicolumn{1}{c} {$ \varphi_{4}$} &
\multicolumn{1}{c} {$ \varphi_{5}$} &
\multicolumn{1}{c} {$ \varphi_{6}$} &
\multicolumn{1}{c} {$ \varphi_{7}$} &
\multicolumn{1}{c} {$ \varphi_{8}$} & 
\multicolumn{1}{c} {$ \varphi_{9}$} & 
\multicolumn{1}{c} {$ \varphi_{10}$} \\
\cline{3-12}
& now & $\LTLalwaysbeen$ & $\rightarrow$ & $S$ & $\LTLonce$ & $r$ & $q$ & $\rightarrow$ & $\bot$ & $\bot$ & $\bot$ \\
\cline{3-12}
\end{tabular}\\
\\
\\
Element $now[7]$ corresponds to an implies formula, with $now[8]$ as the left operand, and $now[9]$ as the right.  The element is assigned $\top$ according to rule 14 because the left operand is $\bot$.\\
\\
Registers after:

\begin{tabular}{rr|c|c|c|c|c|c|c|c|c|c|} &
\multicolumn{1}{c}{} &
\multicolumn{1}{c} {$ \varphi_{1}$} &
\multicolumn{1}{c} {$ \varphi_{2}$} &
\multicolumn{1}{c} {$ \varphi_{3}$} &
\multicolumn{1}{c} {$ \varphi_{4}$} &
\multicolumn{1}{c} {$ \varphi_{5}$} &
\multicolumn{1}{c} {$ \varphi_{6}$} &
\multicolumn{1}{c} {$ \varphi_{7}$} &
\multicolumn{1}{c} {$ \varphi_{8}$} & 
\multicolumn{1}{c} {$ \varphi_{9}$} & 
\multicolumn{1}{c} {$ \varphi_{10}$} \\
\cline{3-12}
& previous & $\top$ & $\top$ & $\top$ & $\top$ & $\bot$ & $\top$ & $\top$ & $\top$ & $\top$ & $\bot$ \\
\cline{3-12}
\\
\multicolumn{1}{c}{} &
\multicolumn{1}{c}{} &
\multicolumn{1}{c} {$ \varphi_{1}$} &
\multicolumn{1}{c} {$ \varphi_{2}$} &
\multicolumn{1}{c} {$ \varphi_{3}$} &
\multicolumn{1}{c} {$ \varphi_{4}$} &
\multicolumn{1}{c} {$ \varphi_{5}$} &
\multicolumn{1}{c} {$ \varphi_{6}$} &
\multicolumn{1}{c} {$ \varphi_{7}$} &
\multicolumn{1}{c} {$ \varphi_{8}$} & 
\multicolumn{1}{c} {$ \varphi_{9}$} & 
\multicolumn{1}{c} {$ \varphi_{10}$} \\
\cline{3-12}
& now & $\LTLalwaysbeen$ & $\rightarrow$ & $S$ & $\LTLonce$ & $r$ & $q$ & $\top$ & $\bot$ & $\bot$ & $\bot$ \\
\cline{3-12}
\end{tabular}\\
\\
\\
\subitem \underline{Event 3, Iteration 5: $t = \langle p, q, r \rangle, e = r, i = 6$}\\
\\
Registers before:

\begin{tabular}{rr|c|c|c|c|c|c|c|c|c|c|} &
\multicolumn{1}{c}{} &
\multicolumn{1}{c} {$ \varphi_{1}$} &
\multicolumn{1}{c} {$ \varphi_{2}$} &
\multicolumn{1}{c} {$ \varphi_{3}$} &
\multicolumn{1}{c} {$ \varphi_{4}$} &
\multicolumn{1}{c} {$ \varphi_{5}$} &
\multicolumn{1}{c} {$ \varphi_{6}$} &
\multicolumn{1}{c} {$ \varphi_{7}$} &
\multicolumn{1}{c} {$ \varphi_{8}$} & 
\multicolumn{1}{c} {$ \varphi_{9}$} & 
\multicolumn{1}{c} {$ \varphi_{10}$} \\
\cline{3-12}
& previous & $\top$ & $\top$ & $\top$ & $\top$ & $\bot$ & $\top$ & $\top$ & $\top$ & $\top$ & $\bot$ \\
\cline{3-12}
\\
\multicolumn{1}{c}{} &
\multicolumn{1}{c}{} &
\multicolumn{1}{c} {$ \varphi_{1}$} &
\multicolumn{1}{c} {$ \varphi_{2}$} &
\multicolumn{1}{c} {$ \varphi_{3}$} &
\multicolumn{1}{c} {$ \varphi_{4}$} &
\multicolumn{1}{c} {$ \varphi_{5}$} &
\multicolumn{1}{c} {$ \varphi_{6}$} &
\multicolumn{1}{c} {$ \varphi_{7}$} &
\multicolumn{1}{c} {$ \varphi_{8}$} & 
\multicolumn{1}{c} {$ \varphi_{9}$} & 
\multicolumn{1}{c} {$ \varphi_{10}$} \\
\cline{3-12}
& now & $\LTLalwaysbeen$ & $\rightarrow$ & $S$ & $\LTLonce$ & $r$ & $q$ & $\top$ & $\bot$ & $\bot$ & $\bot$ \\
\cline{3-12}
\end{tabular}\\
\\
\\
Element $now[6]$ is a literal formula $q$.  It does not match the event $r$, so by rule 10, it is assigned $\bot$.\\
\\
\newpage
Registers after:

\begin{tabular}{rr|c|c|c|c|c|c|c|c|c|c|} &
\multicolumn{1}{c}{} &
\multicolumn{1}{c} {$ \varphi_{1}$} &
\multicolumn{1}{c} {$ \varphi_{2}$} &
\multicolumn{1}{c} {$ \varphi_{3}$} &
\multicolumn{1}{c} {$ \varphi_{4}$} &
\multicolumn{1}{c} {$ \varphi_{5}$} &
\multicolumn{1}{c} {$ \varphi_{6}$} &
\multicolumn{1}{c} {$ \varphi_{7}$} &
\multicolumn{1}{c} {$ \varphi_{8}$} & 
\multicolumn{1}{c} {$ \varphi_{9}$} & 
\multicolumn{1}{c} {$ \varphi_{10}$} \\
\cline{3-12}
& previous & $\top$ & $\top$ & $\top$ & $\top$ & $\bot$ & $\top$ & $\top$ & $\top$ & $\top$ & $\bot$ \\
\cline{3-12}
\\
\multicolumn{1}{c}{} &
\multicolumn{1}{c}{} &
\multicolumn{1}{c} {$ \varphi_{1}$} &
\multicolumn{1}{c} {$ \varphi_{2}$} &
\multicolumn{1}{c} {$ \varphi_{3}$} &
\multicolumn{1}{c} {$ \varphi_{4}$} &
\multicolumn{1}{c} {$ \varphi_{5}$} &
\multicolumn{1}{c} {$ \varphi_{6}$} &
\multicolumn{1}{c} {$ \varphi_{7}$} &
\multicolumn{1}{c} {$ \varphi_{8}$} & 
\multicolumn{1}{c} {$ \varphi_{9}$} & 
\multicolumn{1}{c} {$ \varphi_{10}$} \\
\cline{3-12}
& now & $\LTLalwaysbeen$ & $\rightarrow$ & $S$ & $\LTLonce$ & $r$ & $\bot$ & $\top$ & $\bot$ & $\bot$ & $\bot$ \\
\cline{3-12}
\end{tabular}\\
\\
\\
\subitem \underline{Event 3, Iteration 6: $t = \langle p, q, r \rangle, e = r, i = 5$}\\
\\
Registers before:

\begin{tabular}{rr|c|c|c|c|c|c|c|c|c|c|} &
\multicolumn{1}{c}{} &
\multicolumn{1}{c} {$ \varphi_{1}$} &
\multicolumn{1}{c} {$ \varphi_{2}$} &
\multicolumn{1}{c} {$ \varphi_{3}$} &
\multicolumn{1}{c} {$ \varphi_{4}$} &
\multicolumn{1}{c} {$ \varphi_{5}$} &
\multicolumn{1}{c} {$ \varphi_{6}$} &
\multicolumn{1}{c} {$ \varphi_{7}$} &
\multicolumn{1}{c} {$ \varphi_{8}$} & 
\multicolumn{1}{c} {$ \varphi_{9}$} & 
\multicolumn{1}{c} {$ \varphi_{10}$} \\
\cline{3-12}
& previous & $\top$ & $\top$ & $\top$ & $\top$ & $\bot$ & $\top$ & $\top$ & $\top$ & $\top$ & $\bot$ \\
\cline{3-12}
\\
\multicolumn{1}{c}{} &
\multicolumn{1}{c}{} &
\multicolumn{1}{c} {$ \varphi_{1}$} &
\multicolumn{1}{c} {$ \varphi_{2}$} &
\multicolumn{1}{c} {$ \varphi_{3}$} &
\multicolumn{1}{c} {$ \varphi_{4}$} &
\multicolumn{1}{c} {$ \varphi_{5}$} &
\multicolumn{1}{c} {$ \varphi_{6}$} &
\multicolumn{1}{c} {$ \varphi_{7}$} &
\multicolumn{1}{c} {$ \varphi_{8}$} & 
\multicolumn{1}{c} {$ \varphi_{9}$} & 
\multicolumn{1}{c} {$ \varphi_{10}$} \\
\cline{3-12}
& now & $\LTLalwaysbeen$ & $\rightarrow$ & $S$ & $\LTLonce$ & $r$ & $\bot$ & $\top$ & $\bot$ & $\bot$ & $\bot$ \\
\cline{3-12}
\end{tabular}\\
\\
\\
Element $now[5]$ is the literal formula $r$.  It does match the event being evaluated so the element is assigned $\top$ by rule 10.\\
\\
Registers after:

\begin{tabular}{rr|c|c|c|c|c|c|c|c|c|c|} &
\multicolumn{1}{c}{} &
\multicolumn{1}{c} {$ \varphi_{1}$} &
\multicolumn{1}{c} {$ \varphi_{2}$} &
\multicolumn{1}{c} {$ \varphi_{3}$} &
\multicolumn{1}{c} {$ \varphi_{4}$} &
\multicolumn{1}{c} {$ \varphi_{5}$} &
\multicolumn{1}{c} {$ \varphi_{6}$} &
\multicolumn{1}{c} {$ \varphi_{7}$} &
\multicolumn{1}{c} {$ \varphi_{8}$} & 
\multicolumn{1}{c} {$ \varphi_{9}$} & 
\multicolumn{1}{c} {$ \varphi_{10}$} \\
\cline{3-12}
& previous & $\top$ & $\top$ & $\top$ & $\top$ & $\bot$ & $\top$ & $\top$ & $\top$ & $\top$ & $\bot$ \\
\cline{3-12}
\\
\multicolumn{1}{c}{} &
\multicolumn{1}{c}{} &
\multicolumn{1}{c} {$ \varphi_{1}$} &
\multicolumn{1}{c} {$ \varphi_{2}$} &
\multicolumn{1}{c} {$ \varphi_{3}$} &
\multicolumn{1}{c} {$ \varphi_{4}$} &
\multicolumn{1}{c} {$ \varphi_{5}$} &
\multicolumn{1}{c} {$ \varphi_{6}$} &
\multicolumn{1}{c} {$ \varphi_{7}$} &
\multicolumn{1}{c} {$ \varphi_{8}$} & 
\multicolumn{1}{c} {$ \varphi_{9}$} & 
\multicolumn{1}{c} {$ \varphi_{10}$} \\
\cline{3-12}
& now & $\LTLalwaysbeen$ & $\rightarrow$ & $S$ & $\LTLonce$ & $\top$ & $\bot$ & $\top$ & $\bot$ & $\bot$ & $\bot$ \\
\cline{3-12}
\end{tabular}\\
\\
\\
\subitem \underline{Event 3, Iteration 7: $t = \langle p, q, r \rangle, e = r, i = 4, j = 7$}\\
\\
Registers before:

\begin{tabular}{rr|c|c|c|c|c|c|c|c|c|c|} &
\multicolumn{1}{c}{} &
\multicolumn{1}{c} {$ \varphi_{1}$} &
\multicolumn{1}{c} {$ \varphi_{2}$} &
\multicolumn{1}{c} {$ \varphi_{3}$} &
\multicolumn{1}{c} {$ \varphi_{4}$} &
\multicolumn{1}{c} {$ \varphi_{5}$} &
\multicolumn{1}{c} {$ \varphi_{6}$} &
\multicolumn{1}{c} {$ \varphi_{7}$} &
\multicolumn{1}{c} {$ \varphi_{8}$} & 
\multicolumn{1}{c} {$ \varphi_{9}$} & 
\multicolumn{1}{c} {$ \varphi_{10}$} \\
\cline{3-12}
& previous & $\top$ & $\top$ & $\top$ & $\top$ & $\bot$ & $\top$ & $\top$ & $\top$ & $\top$ & $\bot$ \\
\cline{3-12}\\
\multicolumn{1}{c}{} &
\multicolumn{1}{c}{} &
\multicolumn{1}{c} {$ \varphi_{1}$} &
\multicolumn{1}{c} {$ \varphi_{2}$} &
\multicolumn{1}{c} {$ \varphi_{3}$} &
\multicolumn{1}{c} {$ \varphi_{4}$} &
\multicolumn{1}{c} {$ \varphi_{5}$} &
\multicolumn{1}{c} {$ \varphi_{6}$} &
\multicolumn{1}{c} {$ \varphi_{7}$} &
\multicolumn{1}{c} {$ \varphi_{8}$} & 
\multicolumn{1}{c} {$ \varphi_{9}$} & 
\multicolumn{1}{c} {$ \varphi_{10}$} \\
\cline{3-12}
& now & $\LTLalwaysbeen$ & $\rightarrow$ & $S$ & $\LTLonce$ & $\top$ & $\bot$ & $\top$ & $\bot$ & $\bot$ & $\bot$ \\
\cline{3-12}
\end{tabular}\\
\\
\\
Element $now[4]$ corresponds to a formula with the once operator at its root.  By rule 17 the element is assigned $\top$ \\
\\
Registers after:

\begin{tabular}{rr|c|c|c|c|c|c|c|c|c|c|} &
\multicolumn{1}{c}{} &
\multicolumn{1}{c} {$ \varphi_{1}$} &
\multicolumn{1}{c} {$ \varphi_{2}$} &
\multicolumn{1}{c} {$ \varphi_{3}$} &
\multicolumn{1}{c} {$ \varphi_{4}$} &
\multicolumn{1}{c} {$ \varphi_{5}$} &
\multicolumn{1}{c} {$ \varphi_{6}$} &
\multicolumn{1}{c} {$ \varphi_{7}$} &
\multicolumn{1}{c} {$ \varphi_{8}$} & 
\multicolumn{1}{c} {$ \varphi_{9}$} & 
\multicolumn{1}{c} {$ \varphi_{10}$} \\
\cline{3-12}
& previous & $\top$ & $\top$ & $\top$ & $\top$ & $\bot$ & $\top$ & $\top$ & $\top$ & $\top$ & $\bot$ \\
\cline{3-12}
\\
\multicolumn{1}{c}{} &
\multicolumn{1}{c}{} &
\multicolumn{1}{c} {$ \varphi_{1}$} &
\multicolumn{1}{c} {$ \varphi_{2}$} &
\multicolumn{1}{c} {$ \varphi_{3}$} &
\multicolumn{1}{c} {$ \varphi_{4}$} &
\multicolumn{1}{c} {$ \varphi_{5}$} &
\multicolumn{1}{c} {$ \varphi_{6}$} &
\multicolumn{1}{c} {$ \varphi_{7}$} &
\multicolumn{1}{c} {$ \varphi_{8}$} & 
\multicolumn{1}{c} {$ \varphi_{9}$} & 
\multicolumn{1}{c} {$ \varphi_{10}$} \\
\cline{3-12}
& now & $\LTLalwaysbeen$ & $\rightarrow$ & $S$ & $\top$ & $\top$ & $\bot$ & $\top$ & $\bot$ & $\bot$ & $\bot$ \\
\cline{3-12}
\end{tabular}\\
\\
\\
\subitem \underline{Event 3, Iteration 8: $t = \langle p, q, r \rangle, e = r, i = 3, j = 5, k = 6$}\\
\\
Registers before:

\begin{tabular}{rr|c|c|c|c|c|c|c|c|c|c|} &
\multicolumn{1}{c}{} &
\multicolumn{1}{c} {$ \varphi_{1}$} &
\multicolumn{1}{c} {$ \varphi_{2}$} &
\multicolumn{1}{c} {$ \varphi_{3}$} &
\multicolumn{1}{c} {$ \varphi_{4}$} &
\multicolumn{1}{c} {$ \varphi_{5}$} &
\multicolumn{1}{c} {$ \varphi_{6}$} &
\multicolumn{1}{c} {$ \varphi_{7}$} &
\multicolumn{1}{c} {$ \varphi_{8}$} & 
\multicolumn{1}{c} {$ \varphi_{9}$} & 
\multicolumn{1}{c} {$ \varphi_{10}$} \\
\cline{3-12}
& previous & $\top$ & $\top$ & $\top$ & $\top$ & $\bot$ & $\top$ & $\top$ & $\top$ & $\top$ & $\bot$ \\
\cline{3-12}
\\
\multicolumn{1}{c}{} &
\multicolumn{1}{c}{} &
\multicolumn{1}{c} {$ \varphi_{1}$} &
\multicolumn{1}{c} {$ \varphi_{2}$} &
\multicolumn{1}{c} {$ \varphi_{3}$} &
\multicolumn{1}{c} {$ \varphi_{4}$} &
\multicolumn{1}{c} {$ \varphi_{5}$} &
\multicolumn{1}{c} {$ \varphi_{6}$} &
\multicolumn{1}{c} {$ \varphi_{7}$} &
\multicolumn{1}{c} {$ \varphi_{8}$} & 
\multicolumn{1}{c} {$ \varphi_{9}$} & 
\multicolumn{1}{c} {$ \varphi_{10}$} \\
\cline{3-12}
& now & $\LTLalwaysbeen$ & $\rightarrow$ & $S$ & $\top$ & $\top$ & $\bot$ & $\top$ & $\bot$ & $\bot$ & $\bot$ \\
\cline{3-12}
\end{tabular}\\
\\
\\
The root operator of the $now[3]$ element formula is a since operator.  Element $now[5]$ is the left operand and $now[6]$ is the right operand.  The since operator semantics are defined by rule 18 of Definition \ref{def:AlgorithmicPastOperatorSemantics}.  Element $now[3]$ is assigned $\top$ according to rule 18 because the left operand is $\top$ and the whole formula is $\top$ in $previous[3]$, indicating it was $\top$ for the previous event.\\
\\
Registers after:

\begin{tabular}{rr|c|c|c|c|c|c|c|c|c|c|} &
\multicolumn{1}{c}{} &
\multicolumn{1}{c} {$ \varphi_{1}$} &
\multicolumn{1}{c} {$ \varphi_{2}$} &
\multicolumn{1}{c} {$ \varphi_{3}$} &
\multicolumn{1}{c} {$ \varphi_{4}$} &
\multicolumn{1}{c} {$ \varphi_{5}$} &
\multicolumn{1}{c} {$ \varphi_{6}$} &
\multicolumn{1}{c} {$ \varphi_{7}$} &
\multicolumn{1}{c} {$ \varphi_{8}$} & 
\multicolumn{1}{c} {$ \varphi_{9}$} & 
\multicolumn{1}{c} {$ \varphi_{10}$} \\
\cline{3-12}
& previous & $\top$ & $\top$ & $\top$ & $\top$ & $\bot$ & $\top$ & $\top$ & $\top$ & $\top$ & $\bot$ \\
\cline{3-12}
\\
\multicolumn{1}{c}{} &
\multicolumn{1}{c}{} &
\multicolumn{1}{c} {$ \varphi_{1}$} &
\multicolumn{1}{c} {$ \varphi_{2}$} &
\multicolumn{1}{c} {$ \varphi_{3}$} &
\multicolumn{1}{c} {$ \varphi_{4}$} &
\multicolumn{1}{c} {$ \varphi_{5}$} &
\multicolumn{1}{c} {$ \varphi_{6}$} &
\multicolumn{1}{c} {$ \varphi_{7}$} &
\multicolumn{1}{c} {$ \varphi_{8}$} & 
\multicolumn{1}{c} {$ \varphi_{9}$} & 
\multicolumn{1}{c} {$ \varphi_{10}$} \\
\cline{3-12}
& now & $\LTLalwaysbeen$ & $\rightarrow$ & $\top$ & $\top$ & $\top$ & $\bot$ & $\top$ & $\bot$ & $\bot$ & $\bot$ \\
\cline{3-12}
\end{tabular}\\
\\
\\
\newpage
\subitem \underline{Event 3, Iteration 9: $t = \langle p, q, r \rangle, e = r, i = 2, j = 3, k = 4$}\\
\\
Registers before:

\begin{tabular}{rr|c|c|c|c|c|c|c|c|c|c|} &
\multicolumn{1}{c}{} &
\multicolumn{1}{c} {$ \varphi_{1}$} &
\multicolumn{1}{c} {$ \varphi_{2}$} &
\multicolumn{1}{c} {$ \varphi_{3}$} &
\multicolumn{1}{c} {$ \varphi_{4}$} &
\multicolumn{1}{c} {$ \varphi_{5}$} &
\multicolumn{1}{c} {$ \varphi_{6}$} &
\multicolumn{1}{c} {$ \varphi_{7}$} &
\multicolumn{1}{c} {$ \varphi_{8}$} & 
\multicolumn{1}{c} {$ \varphi_{9}$} & 
\multicolumn{1}{c} {$ \varphi_{10}$} \\
\cline{3-12}
& previous & $\top$ & $\top$ & $\top$ & $\top$ & $\bot$ & $\top$ & $\top$ & $\top$ & $\top$ & $\bot$ \\
\cline{3-12}
\\
\multicolumn{1}{c}{} &
\multicolumn{1}{c}{} &
\multicolumn{1}{c} {$ \varphi_{1}$} &
\multicolumn{1}{c} {$ \varphi_{2}$} &
\multicolumn{1}{c} {$ \varphi_{3}$} &
\multicolumn{1}{c} {$ \varphi_{4}$} &
\multicolumn{1}{c} {$ \varphi_{5}$} &
\multicolumn{1}{c} {$ \varphi_{6}$} &
\multicolumn{1}{c} {$ \varphi_{7}$} &
\multicolumn{1}{c} {$ \varphi_{8}$} & 
\multicolumn{1}{c} {$ \varphi_{9}$} & 
\multicolumn{1}{c} {$ \varphi_{10}$} \\
\cline{3-12}
& now & $\LTLalwaysbeen$ & $\rightarrow$ & $\top$ & $\top$ & $\top$ & $\bot$ & $\top$ & $\bot$ & $\bot$ & $\bot$ \\
\cline{3-12}
\end{tabular}\\
\\
\\
Element $now[2]$ corresponds to an implies formula.  It is assigned $\top$ because the left operand, $now[3]$, and the right operand, $now[4]$, are both $\top$.\\
\\
Registers after:

\begin{tabular}{rr|c|c|c|c|c|c|c|c|c|c|} &
\multicolumn{1}{c}{} &
\multicolumn{1}{c} {$ \varphi_{1}$} &
\multicolumn{1}{c} {$ \varphi_{2}$} &
\multicolumn{1}{c} {$ \varphi_{3}$} &
\multicolumn{1}{c} {$ \varphi_{4}$} &
\multicolumn{1}{c} {$ \varphi_{5}$} &
\multicolumn{1}{c} {$ \varphi_{6}$} &
\multicolumn{1}{c} {$ \varphi_{7}$} &
\multicolumn{1}{c} {$ \varphi_{8}$} & 
\multicolumn{1}{c} {$ \varphi_{9}$} & 
\multicolumn{1}{c} {$ \varphi_{10}$} \\
\cline{3-12}
& previous & $\top$ & $\top$ & $\top$ & $\top$ & $\bot$ & $\top$ & $\top$ & $\top$ & $\top$ & $\bot$ \\
\cline{3-12}\\
\multicolumn{1}{c}{} &
\multicolumn{1}{c}{} &
\multicolumn{1}{c} {$ \varphi_{1}$} &
\multicolumn{1}{c} {$ \varphi_{2}$} &
\multicolumn{1}{c} {$ \varphi_{3}$} &
\multicolumn{1}{c} {$ \varphi_{4}$} &
\multicolumn{1}{c} {$ \varphi_{5}$} &
\multicolumn{1}{c} {$ \varphi_{6}$} &
\multicolumn{1}{c} {$ \varphi_{7}$} &
\multicolumn{1}{c} {$ \varphi_{8}$} & 
\multicolumn{1}{c} {$ \varphi_{9}$} & 
\multicolumn{1}{c} {$ \varphi_{10}$} \\
\cline{3-12}
& now & $\LTLalwaysbeen$ & $\top$ & $\top$ & $\top$ & $\top$ & $\bot$ & $\top$ & $\bot$ & $\bot$ & $\bot$ \\
\cline{3-12}
\end{tabular}\\
\\
\\
\subitem \underline{Event 3, Iteration 10: $t = \langle p, q, r \rangle, e = r, i = 1, j = 2$}\\
\\
Registers before:

\begin{tabular}{rr|c|c|c|c|c|c|c|c|c|c|} &
\multicolumn{1}{c}{} &
\multicolumn{1}{c} {$ \varphi_{1}$} &
\multicolumn{1}{c} {$ \varphi_{2}$} &
\multicolumn{1}{c} {$ \varphi_{3}$} &
\multicolumn{1}{c} {$ \varphi_{4}$} &
\multicolumn{1}{c} {$ \varphi_{5}$} &
\multicolumn{1}{c} {$ \varphi_{6}$} &
\multicolumn{1}{c} {$ \varphi_{7}$} &
\multicolumn{1}{c} {$ \varphi_{8}$} & 
\multicolumn{1}{c} {$ \varphi_{9}$} & 
\multicolumn{1}{c} {$ \varphi_{10}$} \\
\cline{3-12}
& previous & $\top$ & $\top$ & $\top$ & $\top$ & $\bot$ & $\top$ & $\top$ & $\top$ & $\top$ & $\bot$ \\
\cline{3-12}\\
\multicolumn{1}{c}{} &
\multicolumn{1}{c}{} &
\multicolumn{1}{c} {$ \varphi_{1}$} &
\multicolumn{1}{c} {$ \varphi_{2}$} &
\multicolumn{1}{c} {$ \varphi_{3}$} &
\multicolumn{1}{c} {$ \varphi_{4}$} &
\multicolumn{1}{c} {$ \varphi_{5}$} &
\multicolumn{1}{c} {$ \varphi_{6}$} &
\multicolumn{1}{c} {$ \varphi_{7}$} &
\multicolumn{1}{c} {$ \varphi_{8}$} & 
\multicolumn{1}{c} {$ \varphi_{9}$} & 
\multicolumn{1}{c} {$ \varphi_{10}$} \\
\cline{3-12}
& now & $\LTLalwaysbeen$ & $\top$ & $\top$ & $\top$ & $\top$ & $\bot$ & $\top$ & $\bot$ & $\bot$ & $\bot$ \\
\cline{3-12}
\end{tabular}\\
\\
\\
The last element to be evaluated is $now[1]$, it corresponds to a has always been operator and is assigned $\top$ according to rule 16 because the left operand $now[2]$ is $\top$ and the formula was $\top$ for the previous event because $previous[1]$ is $\top$.\\
\\
\newpage
Registers after:

\begin{tabular}{rr|c|c|c|c|c|c|c|c|c|c|} &
\multicolumn{1}{c}{} &
\multicolumn{1}{c} {$ \varphi_{1}$} &
\multicolumn{1}{c} {$ \varphi_{2}$} &
\multicolumn{1}{c} {$ \varphi_{3}$} &
\multicolumn{1}{c} {$ \varphi_{4}$} &
\multicolumn{1}{c} {$ \varphi_{5}$} &
\multicolumn{1}{c} {$ \varphi_{6}$} &
\multicolumn{1}{c} {$ \varphi_{7}$} &
\multicolumn{1}{c} {$ \varphi_{8}$} & 
\multicolumn{1}{c} {$ \varphi_{9}$} & 
\multicolumn{1}{c} {$ \varphi_{10}$} \\
\cline{3-12}
& previous & $\top$ & $\top$ & $\top$ & $\top$ & $\bot$ & $\top$ & $\top$ & $\top$ & $\top$ & $\bot$ \\
\cline{3-12}
\\
\multicolumn{1}{c}{} &
\multicolumn{1}{c}{} &
\multicolumn{1}{c} {$ \varphi_{1}$} &
\multicolumn{1}{c} {$ \varphi_{2}$} &
\multicolumn{1}{c} {$ \varphi_{3}$} &
\multicolumn{1}{c} {$ \varphi_{4}$} &
\multicolumn{1}{c} {$ \varphi_{5}$} &
\multicolumn{1}{c} {$ \varphi_{6}$} &
\multicolumn{1}{c} {$ \varphi_{7}$} &
\multicolumn{1}{c} {$ \varphi_{8}$} & 
\multicolumn{1}{c} {$ \varphi_{9}$} & 
\multicolumn{1}{c} {$ \varphi_{10}$} \\
\cline{3-12}
& now & $\top$ & $\top$ & $\top$ & $\top$ & $\top$ & $\bot$ & $\top$ & $\bot$ & $\bot$ & $\bot$ \\
\cline{3-12}
\end{tabular}\\
\\
\\
\textbf{\item Evaluation Phase - Step 2 repeated}\\
\\
Register before:\\

\begin{tabular}{rr|c|c|c|c|c|c|c|c|c|c|} &
\multicolumn{1}{c}{} &
\multicolumn{1}{c} {$ \varphi_{1}$} &
\multicolumn{1}{c} {$ \varphi_{2}$} &
\multicolumn{1}{c} {$ \varphi_{3}$} &
\multicolumn{1}{c} {$ \varphi_{4}$} &
\multicolumn{1}{c} {$ \varphi_{5}$} &
\multicolumn{1}{c} {$ \varphi_{6}$} &
\multicolumn{1}{c} {$ \varphi_{7}$} &
\multicolumn{1}{c} {$ \varphi_{8}$} & 
\multicolumn{1}{c} {$ \varphi_{9}$} & 
\multicolumn{1}{c} {$ \varphi_{10}$} \\
\cline{3-12}
& previous & $\top$ & $\top$ & $\top$ & $\top$ & $\bot$ & $\top$ & $\top$ & $\top$ & $\top$ & $\bot$ \\
\cline{3-12}
\\
\multicolumn{1}{c}{} &
\multicolumn{1}{c}{} &
\multicolumn{1}{c} {$ \varphi_{1}$} &
\multicolumn{1}{c} {$ \varphi_{2}$} &
\multicolumn{1}{c} {$ \varphi_{3}$} &
\multicolumn{1}{c} {$ \varphi_{4}$} &
\multicolumn{1}{c} {$ \varphi_{5}$} &
\multicolumn{1}{c} {$ \varphi_{6}$} &
\multicolumn{1}{c} {$ \varphi_{7}$} &
\multicolumn{1}{c} {$ \varphi_{8}$} & 
\multicolumn{1}{c} {$ \varphi_{9}$} & 
\multicolumn{1}{c} {$ \varphi_{10}$} \\
\cline{3-12}
& now & $\top$ & $\top$ & $\top$ & $\top$ & $\top$ & $\bot$ & $\top$ & $\bot$ & $\bot$ & $\bot$ \\
\cline{3-12}
\end{tabular}\\
\\
\\
The $r$ event has been completely evaluated so step 1 of evaluation is finished.  Step 2 is to overwrite all the values in the \textit{previous} register with those from the \textit{now} register.\\
\\
Registers after:

\begin{tabular}{rr|c|c|c|c|c|c|c|c|c|c|} &
\multicolumn{1}{c}{} &
\multicolumn{1}{c} {$ \varphi_{1}$} &
\multicolumn{1}{c} {$ \varphi_{2}$} &
\multicolumn{1}{c} {$ \varphi_{3}$} &
\multicolumn{1}{c} {$ \varphi_{4}$} &
\multicolumn{1}{c} {$ \varphi_{5}$} &
\multicolumn{1}{c} {$ \varphi_{6}$} &
\multicolumn{1}{c} {$ \varphi_{7}$} &
\multicolumn{1}{c} {$ \varphi_{8}$} & 
\multicolumn{1}{c} {$ \varphi_{9}$} & 
\multicolumn{1}{c} {$ \varphi_{10}$} \\
\cline{3-12}
& previous & $\top$ & $\top$ & $\top$ & $\top$ & $\top$ & $\bot$ & $\top$ & $\bot$ & $\bot$ & $\bot$ \\
\cline{3-12}
\\
\multicolumn{1}{c}{} &
\multicolumn{1}{c}{} &
\multicolumn{1}{c} {$ \varphi_{1}$} &
\multicolumn{1}{c} {$ \varphi_{2}$} &
\multicolumn{1}{c} {$ \varphi_{3}$} &
\multicolumn{1}{c} {$ \varphi_{4}$} &
\multicolumn{1}{c} {$ \varphi_{5}$} &
\multicolumn{1}{c} {$ \varphi_{6}$} &
\multicolumn{1}{c} {$ \varphi_{7}$} &
\multicolumn{1}{c} {$ \varphi_{8}$} & 
\multicolumn{1}{c} {$ \varphi_{9}$} & 
\multicolumn{1}{c} {$ \varphi_{10}$} \\
\cline{3-12}
& now & $\top$ & $\top$ & $\top$ & $\top$ & $\top$ & $\bot$ & $\top$ & $\bot$ & $\bot$ & $\bot$ \\
\cline{3-12}
\end{tabular}\\\\
\\
\textbf{\item Evaluation Phase - Step 3 repeated}\\\\All events have been evaluated so execution of the algorithm progresses to step 4 immediately.\\
\\
\\
\newpage
\textbf{\item Evaluation Phase - Step 4}\\\\At this point, all events in the trace have been evaluated and the satisfaction of formula $\varphi$ over trace $t$ has been determined as the value assigned to the element corresponding to root operator of formula $\varphi$.  Satisfaction of the formula is read from the \textit{previous} register rather than the \textit{now} register despite both containing the same assignments.  The reason being, that in the case of an empty trace, the \textit{previous} register will be initialised to the result of evaluating the formula over an empty trace, while elements of the \textit{now} register will not be assigned any values.\\

\subitem \underline{Final state}

\begin{tabular}{rr|c|c|c|c|c|c|c|c|c|c|} &
\multicolumn{1}{c}{} &
\multicolumn{1}{c} {$ \varphi_{1}$} &
\multicolumn{1}{c} {$ \varphi_{2}$} &
\multicolumn{1}{c} {$ \varphi_{3}$} &
\multicolumn{1}{c} {$ \varphi_{4}$} &
\multicolumn{1}{c} {$ \varphi_{5}$} &
\multicolumn{1}{c} {$ \varphi_{6}$} &
\multicolumn{1}{c} {$ \varphi_{7}$} &
\multicolumn{1}{c} {$ \varphi_{8}$} & 
\multicolumn{1}{c} {$ \varphi_{9}$} & 
\multicolumn{1}{c} {$ \varphi_{10}$} \\
\cline{3-12}
& previous & $\top$ & $\top$ & $\top$ & $\top$ & $\top$ & $\bot$ & $\top$ & $\bot$ & $\bot$ & $\bot$ \\
\cline{3-12}
\end{tabular}\\
\\
\\
$ t \models \varphi = previous[1] = \top $

\qed
\end{myEx}
