%% Here you can specify new commands and environments that you intend
%% to use. Using commands can make your document easier to write, read
%% and be more consistent.

%% An example could be
%% \newcommand{\CASL}{\textrm{\textsc{Casl}}\xspace}
%% This would define the command \CASL, that would produce the LaTeX
%% code '\textrm{\textsc{Casl}}' (with an appropriate space at the
%% end) each time it was used.


\newcommand{\CCSig}{\textsc{CspCaslSig}\xspace}
\newcommand{\CCSigN}{\textsc{CspCaslSig}^{plain}\xspace}

\newenvironment{LINES}{\array[t]{@{}l@{}}}{\endarray}


\newcommand{\point}[1]{
    \todo[inline, color=blue!20]{
      $\blacktriangleright$ #1
    }
}

\newcommand{\subpoint}[1]{
    \todo[inline, color=blue!20]{
      $\qquad \vartriangleright$ #1
    }
}

\newcommand{\appendixLemma}[2]{\noindent \textbf{Lemma \ref{#1}} #2}

\newcommand{\XML}{XML\xspace}
\newcommand{\EP}{EP2\xspace}

\newcommand{\ConsExt}{\mathrel{\text{\textit{ConsExt}}}}
\newcommand{\image}{\mathop{\text{\textit{image}}}}

%% WADT 06 paper
\newcommand{\inp}{\mathit{param}}
\newcommand{\comms}{\mathit{comms}}
\newcommand{\denotation}{\mathit{denotation}}

%% General
\newcommand{\CASL}{\textrm{\textsc{Casl}}\xspace}
\newcommand{\CSP}{\textrm{\textsc{Csp}}\xspace}
\newcommand{\CC}{\CSP-\CASL}
\newcommand{\SCC}{Structured \CSP-\CASL}
\newcommand{\CSPProver}{\CSP-Prover\xspace}
\newcommand{\CCProver}{\CC-Prover\xspace}
\newcommand{\Hets}{\textrm{\textsc{Hets}}\xspace}
\newcommand{\SPASS}{SPASS\xspace}
\newcommand{\ML}{ML\xspace}

\newcommand{\Nat}{\mathbb{N}\xspace}

\newcommand{\expandsTo}{\widehat{=}}

%% The def extension on sorts, .e.g., s_def
\newcommand{\sortdef}{\text{\textit{def}}}

\newcommand{\partialInv}[1]{{#1}^{-1}}
\newcommand{\invImg}[1]{{#1}^{-}}

\newcommand{\upCl}[1]{\mathop{\uparrow} #1\xspace}
\newcommand{\downCl}[1]{\mathop{\downarrow} #1\xspace}
\newcommand{\downClSet}[1]{{#1}_{\downarrow}\xspace}
\newcommand{\overloadRelProc}{\sim_{N}}

 %% CC: evaluation according to csp semantics relative to csp domain
\newcommand{\cspSemantics}[2]{\sem{#1}_{#2}\xspace}
 %% CC: evaluation according to casl semantics relative to cc
 %% model, local and global variable valuations.
\newcommand{\ccEvalCaslSem}[4]{\sem{#1}_{#2, #3, #4}\xspace}

%% Related work
\newcommand{\ActOne}{\textrm{\textsc{Act-One}}\xspace}
\newcommand{\CCS}{\textrm{\textsc{Ccs}}\xspace}
\newcommand{\CCSCASL}{\textrm{\textsc{Ccs-Casl}}\xspace}
\newcommand{\CASLCHARTS}{\textrm{\textsc{Casl-Charts}}\xspace}
\newcommand{\COCASL}{\textrm{\textsc{Co-Casl}}\xspace}
\newcommand{\CASLLTL}{\textrm{\textsc{Casl-Ltl}}\xspace}
\newcommand{\Lotos}{\textrm{\textsc{Lotos}}\xspace}
\newcommand{\ELotos}{\textrm{\textsc{ELotos}}\xspace}
\newcommand{\CSPM}{\textrm{\textsc{Csp}}\ensuremath{_{\textrm{M}}}\xspace}
\newcommand{\ACP}{\textrm{\textsc{Acp}}\xspace}
\newcommand{\ASF}{\textrm{\textsc{Asf}}\xspace}
\newcommand{\PSF}{\textrm{\textsc{Psf}}\xspace}
\newcommand{\MUCRL}{\ensuremath{\mu}\textrm{\textsc{Crl}}\xspace}
\newcommand{\MCRL}{\textrm{\textsc{MCrl}}\xspace}
\newcommand{\MUCRLTwo}{\ensuremath{\mu}\textrm{\textsc{Crl2}}\xspace}
\newcommand{\CRL}{\textrm{\textsc{Crl}}\xspace}
\newcommand{\CTLStar}{\ensuremath{\textrm{\textsc{Ctl}}^{*}}\xspace}
\newcommand{\CSPOZ}{\textrm{\textsc{Csp-OZ}}\xspace}
\newcommand{\Circus}{\textrm{\textsc{Circus}}\xspace}
\newcommand{\Eucalyptus}{\textrm{\textsc{Eucalyptus}}\xspace}
\newcommand{\CZT}{\textrm{\textsc{Czt}}\xspace}
\newcommand{\FDR}{\textrm{\textsc{Fdr}}\xspace}
\newcommand{\ProBE}{\textrm{\textsc{ProBE}}\xspace}
\newcommand{\UTP}{\textrm{UTP}\xspace}
\newcommand{\OSI}{\textrm{OSI}\xspace}
\newcommand{\ISO}{\textrm{ISO}\xspace}
\newcommand{\ObjectZ}{\textrm{\textsc{Object-Z}}\xspace}

%% Semantics
\newcommand{\angles}[1]{\langle #1 \rangle}
\newcommand{\concat}{\mathop{{}^{\smallfrown}}}
\newcommand{\T}{\mathcal T}
\newcommand{\F}{\mathcal F}
\newcommand{\N}{\mathcal N}
\newcommand{\R}{\mathcal R}
\newcommand{\D}{\mathcal D}
\newcommand{\failuresdivergences}{Failures/Divergences\xspace}
\newcommand{\stablefailures}{Stable-Failures\xspace}
\newcommand{\stablerevivals}{Stable Revivals\xspace}

\newcommand{\tick}{\cspTick}
\newcommand{\powerset}{\ensuremath{\mathcal P}}
\newcommand{\trBot}{tr_{\bot}}
\newcommand{\restrictedTo}[1]{\upharpoonright #1}
\newcommand{\sem}[1]{\llbracket #1 \rrbracket}
\newcommand{\cspRef}{\sqsubseteq}
\newcommand{\cspRefReversed}{\sqsupseteq}
\newcommand{\ccref}{\leadsto}


\newcommand{\RefCl}{\mathop{\text{\textit{RefCl}}}}

\newcommand{\traces}{\mathop{\text{\textit{traces}}}}
\newcommand{\failures}{\mathop{\text{\textit{failures}}}}
\newcommand{\failuresBot}{\mathop{\text{\textit{failures}}_{\bot}}}
\newcommand{\failuresBotI}{\mathop{\text{\textit{failures}}_{\bot, I}}}
\newcommand{\divergences}{\mathop{\text{\textit{divergences}}}}
\newcommand{\Network}{\mathop{\text{\textit{Network}}}}

\newcommand{\ccalph}{\ensuremath{\mathop{\textit{Alph}}}}
\newcommand{\ccalphReduct}{\ensuremath{\mathop{\textit{Alph\_Reduct}}}}
\newcommand{\alphEmb}[1]{\overline{#1}}
\newcommand{\strip}{\ensuremath{\mathop{\text{\textit{strip}}}}}

\newcommand{\cspTop}{\text{\textit{Top}}}

%% From MPhil
\newcommand{\category}[1]{\ensuremath{\mathop{\text{\textbf{#1}}}}}
\newcommand{\functor}[1]{\ensuremath{\mathop{\text{\textbf{#1}}}}}
\newcommand{\functorsymbol}[1]{#1}

\newcommand{\ModReduct}{\ensuremath{\category{Mod\_Reduct}}\xspace}

% These are institution functors
\newcommand{\Sen}{\functor{sen}}
\newcommand{\ModFunctor}{\functor{mod}}

% these are (projection) functions
\newcommand{\Sig}{\ensuremath{\mathop{\text{\textbf{Sig}}}}}
\newcommand{\Mod}{\ensuremath{\mathop{\text{\textbf{Mod}}}}}
\newcommand{\Axioms}{\ensuremath{\mathop{\text{\textbf{Ax}}}}}
\newcommand{\flatten}{\ensuremath{\mathop{\text{\textit{flatten}}}}}
\newcommand{\LHS}{L.H.S\xspace}
\newcommand{\RHS}{R.H.S\xspace}
\newcommand{\Isabelle}{Isabelle\xspace}
\newcommand{\IsabelleHOL}{Isabelle/HOL\xspace}
\newcommand{\inj}{\mbox{\texttt{inj}}\xspace}
\newcommand{\pr}{\mbox{\texttt{pr}}\xspace}
%% Structuring Abreviations
\newcommand{\spec}[1]{\text{\textsc{#1}}}
\newcommand{\Sand}{\mathrel{\mathbf{and}}}
\newcommand{\Sthen}{\mathrel{\mathbf{then}}}
\newcommand{\Srename}{\mathrel{\mathbf{rename}}}
\newcommand{\Shide}{\mathrel{\mathbf{hide}}}
\newcommand{\Sfree}[2]{\mathop{\mathbf{free}} #1 \mathrel{\mathbf{along}} #2}
\newcommand{\Swith}{\mathrel{\mathbf{with}}}
\newcommand{\FA}{\mathcm{FA}}
\newcommand{\FM}{\mathcm{FM}}

\newcommand{\singleValued}[1]{\mathop{\text{Single-valued}}(#1)}
\newcommand{\procCons}[2]{\mathop{\text{ProcConst}}(#1, #2)}
\newcommand{\isDF}[2]{#1 \, \mathbin{isDFin} \, #2 \xspace}
\newcommand{\resToLiveTickOn}[4]{#1 \mathrel{ResToLive^{\tick}} #2 \mathrel{on} #3 \mathrel{in} #4}

\newcommand{\FOL}{\text{\textit{FOL$^{=}$}}\xspace}
\newcommand{\PFOL}{\text{\textit{PFOL$^{=}$}}\xspace}
\newcommand{\PCFOL}{\text{\textit{PCFOL$^{=}$}}\xspace}
\newcommand{\SubPFOL}{\text{\textit{SubPFOL$^{=}$}}\xspace}
\newcommand{\SubPCFOL}{\text{\textit{SubPCFOL$^{=}$}}\xspace}
\newcommand{\ResSubPCFOL}{\text{\textit{ResSubPCFOL$^{=}$}}\xspace}

\newcommand{\fst}{\mathop{\text{\textit{fst}}}}
\newcommand{\snd}{\mathop{\text{\textit{snd}}}}
\newcommand{\SET}{\category{SET}\xspace}
\newcommand{\SETINJ}{\category{SET\_INJ}\xspace}
\newcommand{\op}{op}
%% The projection of the controlled traces of a csp model
\newcommand{\contTraces}{cTr}

%% Tikz Styles
\tikzstyle{category}     =[circle,draw, minimum size=3cm]
\tikzstyle{categoryLarge}=[ellipse,draw, minimum width=10cm, minimum height=5.5cm]
\tikzstyle{functor}      =[->, above, shorten <=0.2cm, shorten >=0.2cm]
\tikzstyle{object}       =[]
\tikzstyle{morphism}     =[->]


\newenvironment{discussion}
{\begin{Sbox}\begin{minipage}{0.95\textwidth}\textbf{Discussion:}}
{\end{minipage}\end{Sbox}\begin{center}\shadowbox{\TheSbox}\end{center}}


% Haskell listings.
%
% Style for all Haskell code.
%
\lstdefinestyle{haskell}
    {language=Haskell,
      columns=fixed,
      %%columns=fullflexible,
      showstringspaces=false,
      %%numbers=left,
      %%numberstyle=\tiny,
      basicstyle=\small\ttfamily,
      numbersep=5pt,
      %%numberblanklines=false,
      %% breaklines=true,
    }
%
% Environment for long Haskell listings.
%
\lstnewenvironment{HaskellCode}
  {\lstset{style=haskell, frame=single}}
  {}

\newcommand{\inHaskell}[1]{\lstinline[style=haskell]{#1}}

\newcommand{\amalgamate}{\oplus}
\newcommand{\abovearrow}[1]{\overset{\rightarrow}{#1}}

\newcommand{\GRKH}{\text{Grigore Ro\c{s}u and Klaus Havelund}}
\newcommand{\RH}{\text{Ro\c{s}u-Havelund}}
\newcommand{\RRH}{\text{Reverse-Ro\c{s}u-Havelund}}
\newcommand{\Buchi}{\text{B\"{u}chi}}

%LTL symbols
\newcommand{\LTLeventually}{\Diamond}
\newcommand{\LTLnext}{\scalebox{1.4}{$\circ$}}
\newcommand{\LTLalways}{\square}

\newcommand{\LTLonce}{\ensurestackMath{%
  \stackengine{.5pt}{\Diamond}{\scalebox{.6}[1.2]{$-$}}{O}{c}{F}{F}{L}}}

\newcommand{\LTLprevious}{\scalebox{1.4}{\ensurestackMath{%
  \stackengine{.5pt}{\circ}{\scalebox{.4}[.85]{$-$}}{O}{c}{F}{F}{L}}}}

\newcommand{\LTLalwaysbeen}{\ensurestackMath{%
  \stackengine{.5pt}{\square}{\scalebox{.7}[1.2]{$-$}}{O}{c}{F}{F}{L}}}

